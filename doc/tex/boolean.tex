% -------------------------------------------------------------------------
% ------ nuweb macros (redefine as desired, or omit with "nuweb -p") ------
% -------------------------------------------------------------------------
\providecommand{\NWtxtMacroDefBy}{Macro defined by}
\providecommand{\NWtxtMacroRefIn}{Macro referenced in}
\providecommand{\NWtxtMacroNoRef}{Macro never referenced}
\providecommand{\NWtxtDefBy}{Defined by}
\providecommand{\NWtxtRefIn}{Referenced in}
\providecommand{\NWtxtNoRef}{Not referenced}
\providecommand{\NWtxtFileDefBy}{File defined by}
\providecommand{\NWsep}{${\diamond}$}
\providecommand{\NWlink}[2]{\hyperlink{#1}{#2}}
\providecommand{\NWtarget}[2]{% move baseline up by \baselineskip 
  \raisebox{\baselineskip}[1.5ex][0ex]{%
    \mbox{%
      \hypertarget{#1}{%
        \raisebox{-1\baselineskip}[0ex][0ex]{%
          \mbox{#2}%
}}}}}
% -------------------------------------------------------------------------

\documentclass[11pt,oneside]{article}	%use"amsart"insteadof"article"forAMSLaTeXformat
\usepackage{geometry}		%Seegeometry.pdftolearnthelayoutoptions.Therearelots.
\geometry{letterpaper}		%...ora4paperora5paperor...
%\geometry{landscape}		%Activateforforrotatedpagegeometry
%\usepackage[parfill]{parskip}		%Activatetobeginparagraphswithanemptylineratherthananindent
\usepackage{graphicx}				%Usepdf,png,jpg,oreps•withpdflatex;useepsinDVImode
								%TeXwillautomaticallyconverteps-->pdfinpdflatex		
\usepackage{amssymb}
\usepackage{hyperref}

%----macros begin---------------------------------------------------------------
\usepackage{color}
\usepackage{amsthm}

\def\conv{\mbox{\textrm{conv}\,}}
\def\aff{\mbox{\textrm{aff}\,}}
\def\E{\mathbb{E}}
\def\R{\mathbb{R}}
\def\Z{\mathbb{Z}}
\def\tex{\TeX}
\def\latex{\LaTeX}
\def\v#1{{\bf #1}}
\def\p#1{{\bf #1}}
\def\T#1{{\bf #1}}

\def\vet#1{{\left(\begin{array}{cccccccccccccccccccc}#1\end{array}\right)}}
\def\mat#1{{\left(\begin{array}{cccccccccccccccccccc}#1\end{array}\right)}}

\def\lin{\mbox{\rm lin}\,}
\def\aff{\mbox{\rm aff}\,}
\def\pos{\mbox{\rm pos}\,}
\def\cone{\mbox{\rm cone}\,}
\def\conv{\mbox{\rm conv}\,}
\newcommand{\homog}[0]{\mbox{\rm homog}\,}
\newcommand{\relint}[0]{\mbox{\rm relint}\,}

%----macros end-----------------------------------------------------------------

\title{Boolean operations with chains
\footnote{This document is part of the \emph{Linear Algebraic Representation with CoChains} (LAR-CC) framework~\cite{cclar-proj:2013:00}. \today}
}
\author{Alberto Paoluzzi}
%\date{}							%Activatetodisplayagivendateornodate

\begin{document}
\maketitle
%\nonstopmode

%----macros end-----------------------------------------------------------------
%>>>>>>>>>>>>>>>>>>>>>>>>>>>>>>>>>>>>>>>>>>>>>>>>>>>>>>>>>>>>>>>>>>>>>>>>>>>>>>>
\begin{abstract}
Boolean operations are a major addition to every geometric package. Union, intersection, difference and complementation of decomposed spaces are discussed and implemented in this module by making use of the Linear Algebraic Representation (LAR) introduced in~\cite{Dicarlo:2014:TNL:2543138.2543294}. First, the two finite decompositions are merged, by merging their vertices (0-cells of support spaces); then a Delaunay complex based on the vertex set union is computed, and the shared $d$-chain is extracted and splitted, according to the cell structure of the input $d$-chains. The results of a Boolean operation are finally computed by sum, product or difference of the (binary) coordinate representation of the (splitted) argument chains, by using the novel chain-basis resulted from a splitting stage. Differently from the totality of algorithms known to the authors, no search or traversal of some (complicated) data structure is performed by this algorithm. 
\end{abstract}

\tableofcontents
%<<<<<<<<<<<<<<<<<<<<<<<<<<<<<<<<<<<<<<<<<<<<<<<<<<<<<<<<<<<<<<<<<<<<<<<<<<<<<<<
%>>>>>>>>>>>>>>>>>>>>>>>>>>>>>>>>>>>>>>>>>>>>>>>>>>>>>>>>>>>>>>>>>>>>>>>>>>>>>>>
\section{Introduction}

In this section we introduce and shortly outline our novel algorithm for Boolean operations with chain of cells from different space decompositions implemented in this LAR-CC software module.

The input objects are denoted in the remainder as $X_1$ and $X_2$, and their finite cell decompositions as $\Lambda_1$ and $\Lambda_1$. Our goal is to compute $X = X_1\, op\, X_2$, where $op \in \{\cup ,\cap , - ,\ominus \}$ or $\complement X$, based on a common decomposition $\Lambda = \Lambda_1\, op\, \Lambda_2$, with $\Lambda$ being a suitably fragmented decomposition of the X space.

Of course, we aim to compute a minimal (in some sense) decomposition, making the best use of the LAR framework, based on CSR representation of sparse binary matrices and standard matrix algebra operations.

\subsection{User interface}

The API will contain the high-level binary functions \texttt{union}, \texttt{intersection}, \texttt{difference}, and \texttt{xor}. Each of them will call the same function \texttt{boolOps} and then suitably operates the two returned bit arrays, i.e.~the coordinate representations of the input spaces in the merged cell decomposition. The input parameters \texttt{lar1} and  \texttt{lar2} stand for two LAR models, each one constituted by a pair \texttt{(V,CV)}, i.e.~by the matrix \texttt{V} of vertex coordinates and by an integer array \texttt{CV} giving the vertex indices of each $d$-cell.

%-------------------------------------------------------------------------------
\begin{flushleft} \small \label{scrap1}
\protect\makebox[0ex][r]{\NWtarget{nuweb2a}{\rule{0ex}{0ex}}\hspace{1em}}$\langle\,$High-level boolean operations\nobreak\ {\footnotesize 2a}$\,\rangle\equiv$
\vspace{-1ex}
\begin{list}{}{} \item
\mbox{}\verb@def union(lar1,lar2):@\\
\mbox{}\verb@   lar = boolOps(lar1,lar2)@\\
\mbox{}\verb@def intersection(lar1,lar2):@\\
\mbox{}\verb@   lar = boolOps(lar1,lar2)@\\
\mbox{}\verb@def difference(lar1,lar2):@\\
\mbox{}\verb@   lar = boolOps(lar1,lar2)@\\
\mbox{}\verb@def xor(lar1,lar2):@\\
\mbox{}\verb@   lar = boolOps(lar1,lar2)@\\
\mbox{}\verb@@{\NWsep}
\end{list}
\vspace{-1ex}
\footnotesize\addtolength{\baselineskip}{-1ex}
\begin{list}{}{\setlength{\itemsep}{-\parsep}\setlength{\itemindent}{-\leftmargin}}
\item \NWtxtMacroRefIn\ \NWlink{nuweb8}{8}.
\end{list}
\end{flushleft}
%-------------------------------------------------------------------------------

%<<<<<<<<<<<<<<<<<<<<<<<<<<<<<<<<<<<<<<<<<<<<<<<<<<<<<<<<<<<<<<<<<<<<<<<<<<<<<<<
%>>>>>>>>>>>>>>>>>>>>>>>>>>>>>>>>>>>>>>>>>>>>>>>>>>>>>>>>>>>>>>>>>>>>>>>>>>>>>>>
\section{Merging 0-cells}
%<<<<<<<<<<<<<<<<<<<<<<<<<<<<<<<<<<<<<<<<<<<<<<<<<<<<<<<<<<<<<<<<<<<<<<<<<<<<<<<

The real work is performed by the function \texttt{boolOps}, that will procede step-by-step to the computation of the minimally fragmented common cell complex where to compute the chain resulting from the requested Boolean operation.

%-------------------------------------------------------------------------------
\begin{flushleft} \small
\begin{minipage}{\linewidth} \label{scrap2}
\protect\makebox[0ex][r]{\NWtarget{nuweb2b}{\rule{0ex}{0ex}}\hspace{1em}}$\langle\,$Boolean subdivided complex\nobreak\ {\footnotesize 2b}$\,\rangle\equiv$
\vspace{-1ex}
\begin{list}{}{} \item
\mbox{}\verb@""" High level Boolean Application Programming Interface """@\\
\mbox{}\verb@def boolOps(lar1,lar2):@\\
\mbox{}\verb@   V1,CV1 = lar1@\\
\mbox{}\verb@   V2,CV2 = lar2@\\
\mbox{}\verb@   n1,n2 = len(V1),len(V2)@\\
\mbox{}\verb@   V, CV1, CV2, n12 = vertexRenumbering(lar1, lar2)@\\
\mbox{}\verb@@\\
\mbox{}\verb@   CV = Delaunay(array(V)).vertices@\\
\mbox{}\verb@   CV_un, CV_int = splitDelaunayComplex(CV,n1,n2,n12)@\\
\mbox{}\verb@   return V,CV_un, CV_int, n1,n2,n12@\\
\mbox{}\verb@@{\NWsep}
\end{list}
\vspace{-1ex}
\footnotesize\addtolength{\baselineskip}{-1ex}
\begin{list}{}{\setlength{\itemsep}{-\parsep}\setlength{\itemindent}{-\leftmargin}}
\item \NWtxtMacroRefIn\ \NWlink{nuweb8}{8}.
\end{list}
\end{minipage}\\[4ex]
\end{flushleft}
%-------------------------------------------------------------------------------


%-------------------------------------------------------------------------------
\subsection{Global reordering of vertex coordinates}
%-------------------------------------------------------------------------------
A global reordering of vertex coordinates is executed as the first step of the Boolean algorithm, in order to eliminate the duplicate vertices, by substituting double vertex copies (coming from the two close points) with a single instance. 

Two dictionaries are created, then merged in a single dictionary, and finally split into three subsets of (vertex,index) pairs, with the aim of rebuilding the input representations, by making use of a novel and more useful vertex indexing.

The union set of vertices is finally reordered using the three subsets of vertices belonging (a) only to the first argument, (b) only to the second argument and (c) to both, respectively denoted as $V_1, V_2, V_{12}$. A top-down description of this initial computational step is provided by the set of macros discussed in this section.

%-------------------------------------------------------------------------------
\begin{flushleft} \small
\begin{minipage}{\linewidth} \label{scrap3}
\protect\makebox[0ex][r]{\NWtarget{nuweb3a}{\rule{0ex}{0ex}}\hspace{1em}}$\langle\,$Place the vertices of Boolean arguments in a common space\nobreak\ {\footnotesize 3a}$\,\rangle\equiv$
\vspace{-1ex}
\begin{list}{}{} \item
\mbox{}\verb@""" First step of Boolen Algorithm """@\\
\mbox{}\verb@@\hbox{$\langle\,$Initial indexing of vertex positions\nobreak\ {\footnotesize \NWlink{nuweb3b}{3b}}$\,\rangle$}\verb@@\\
\mbox{}\verb@@\hbox{$\langle\,$Merge two dictionaries with keys the point locations\nobreak\ {\footnotesize \NWlink{nuweb4a}{4a}}$\,\rangle$}\verb@@\\
\mbox{}\verb@@\hbox{$\langle\,$Filter the common dictionary into three subsets\nobreak\ {\footnotesize \NWlink{nuweb4b}{4b}}$\,\rangle$}\verb@@\\
\mbox{}\verb@@\hbox{$\langle\,$Compute an inverted index to reorder the vertices of Boolean arguments\nobreak\ {\footnotesize \NWlink{nuweb5a}{5a}}$\,\rangle$}\verb@@\\
\mbox{}\verb@@\hbox{$\langle\,$Return the single reordered pointset and the two $d$-cell arrays\nobreak\ {\footnotesize \NWlink{nuweb5b}{5b}}$\,\rangle$}\verb@@\\
\mbox{}\verb@@{\NWsep}
\end{list}
\vspace{-1ex}
\footnotesize\addtolength{\baselineskip}{-1ex}
\begin{list}{}{\setlength{\itemsep}{-\parsep}\setlength{\itemindent}{-\leftmargin}}
\item \NWtxtMacroRefIn\ \NWlink{nuweb8}{8}.
\end{list}
\end{minipage}\\[4ex]
\end{flushleft}
%-------------------------------------------------------------------------------

%-------------------------------------------------------------------------------
\subsubsection{Re-indexing of vertices}
%-------------------------------------------------------------------------------

\paragraph{Initial indexing of vertex positions}
The input LAR models are located in a common space by (implicitly) joining \texttt{V1} and \texttt{V2} in a same array, and (explicitly) shifting the vertex indices in \texttt{CV2} by the length of \texttt{V1}.
%-------------------------------------------------------------------------------
\begin{flushleft} \small \label{scrap4}
\protect\makebox[0ex][r]{\NWtarget{nuweb3b}{\rule{0ex}{0ex}}\hspace{1em}}$\langle\,$Initial indexing of vertex positions\nobreak\ {\footnotesize 3b}$\,\rangle\equiv$
\vspace{-1ex}
\begin{list}{}{} \item
\mbox{}\verb@from collections import defaultdict, OrderedDict@\\
\mbox{}\verb@@\\
\mbox{}\verb@def vertexRenumbering(model1, model2):@\\
\mbox{}\verb@   V1,CV1 = model1; V2,CV2 = model2@\\
\mbox{}\verb@   n = len(V1); m = len(V2)@\\
\mbox{}\verb@   def shift(CV, n): @\\
\mbox{}\verb@      return [[v+n for v in cell]for cell in CV]@\\
\mbox{}\verb@   CV2 = shift(CV2,n)@\\
\mbox{}\verb@@{\NWsep}
\end{list}
\vspace{-1ex}
\footnotesize\addtolength{\baselineskip}{-1ex}
\begin{list}{}{\setlength{\itemsep}{-\parsep}\setlength{\itemindent}{-\leftmargin}}
\item \NWtxtMacroRefIn\ \NWlink{nuweb3a}{3a}.
\end{list}
\end{flushleft}
%-------------------------------------------------------------------------------

\paragraph{Merge two dictionaries with point location as keys}
Since currently \texttt{CV1} and \texttt{CV2} point to a set of vertices larger than their initial sets 
\texttt{V1} and \texttt{V2}, we index the set $\texttt{V1} \cup \texttt{V2}$ using a Python \texttt{defaultdict} dictionary, in order to avoid errors of "missing key". As dictionary keys, we use the string representation of the vertex position vector provided by the \texttt{vcode} function given in the Appendix.
%-------------------------------------------------------------------------------
\begin{flushleft} \small \label{scrap5}
\protect\makebox[0ex][r]{\NWtarget{nuweb4a}{\rule{0ex}{0ex}}\hspace{1em}}$\langle\,$Merge two dictionaries with keys the point locations\nobreak\ {\footnotesize 4a}$\,\rangle\equiv$
\vspace{-1ex}
\begin{list}{}{} \item
\mbox{}\verb@@\\
\mbox{}\verb@   vdict1 = defaultdict(list)@\\
\mbox{}\verb@   for k,v in enumerate(V1): vdict1[vcode(v)].append(k) @\\
\mbox{}\verb@   vdict2 = defaultdict(list)@\\
\mbox{}\verb@   for k,v in enumerate(V2): vdict2[vcode(v)].append(k+n) @\\
\mbox{}\verb@   @\\
\mbox{}\verb@   vertdict = defaultdict(list)@\\
\mbox{}\verb@   for point in vdict1.keys(): vertdict[point] += vdict1[point]@\\
\mbox{}\verb@   for point in vdict2.keys(): vertdict[point] += vdict2[point]@\\
\mbox{}\verb@@{\NWsep}
\end{list}
\vspace{-1ex}
\footnotesize\addtolength{\baselineskip}{-1ex}
\begin{list}{}{\setlength{\itemsep}{-\parsep}\setlength{\itemindent}{-\leftmargin}}
\item \NWtxtMacroRefIn\ \NWlink{nuweb3a}{3a}.
\end{list}
\end{flushleft}
%-------------------------------------------------------------------------------

\paragraph{Example of string coding of a vertex position}
The position vector of a point of real coordinates is provided by the function \texttt{vcode}.
An example of coding is given below. The \emph{precision} of the string representation can be tuned at will.
{\small
\begin{verbatim}
>>> vcode([-0.011660381062724849, 0.297350056848685860])
'[-0.0116604, 0.2973501]'
\end{verbatim}}



\paragraph{Filter the common dictionary into three subsets}
\texttt{Vertdict}, dictionary of vertices, uses as key stye position vectors of vertices coded as string, and as values the list of integer indices of vertices on the given position. If the point position belongs either to the first or to second argument only, it is stored in \texttt{case1} or \texttt{case2} lists respectively. If the position (\texttt{item.key}) is shared between two vertices, it is stored in \texttt{case12}.
The variables \texttt{n1}, \texttt{n2}, and \texttt{n12} remember the number of vertices respectively stored in each repository.
%-------------------------------------------------------------------------------
\begin{flushleft} \small \label{scrap6}
\protect\makebox[0ex][r]{\NWtarget{nuweb4b}{\rule{0ex}{0ex}}\hspace{1em}}$\langle\,$Filter the common dictionary into three subsets\nobreak\ {\footnotesize 4b}$\,\rangle\equiv$
\vspace{-1ex}
\begin{list}{}{} \item
\mbox{}\verb@@\\
\mbox{}\verb@   case1, case12, case2 = [],[],[]@\\
\mbox{}\verb@   for item in vertdict.items():@\\
\mbox{}\verb@      key,val = item@\\
\mbox{}\verb@      if len(val)==2:  case12 += [item]@\\
\mbox{}\verb@      elif val[0] < n: case1 += [item]@\\
\mbox{}\verb@      else: case2 += [item]@\\
\mbox{}\verb@   n1 = len(case1); n2 = len(case12); n3 = len(case2)@\\
\mbox{}\verb@@{\NWsep}
\end{list}
\vspace{-1ex}
\footnotesize\addtolength{\baselineskip}{-1ex}
\begin{list}{}{\setlength{\itemsep}{-\parsep}\setlength{\itemindent}{-\leftmargin}}
\item \NWtxtMacroRefIn\ \NWlink{nuweb3a}{3a}.
\end{list}
\end{flushleft}
%-------------------------------------------------------------------------------

\paragraph{Compute an inverted index to reorder the vertices of Boolean arguments}
The new indices of vertices are computed according with their position within the storage repositories \texttt{case1}, \texttt{case2}, and \texttt{case12}. Notice that every \texttt{item[1]} stored in \texttt{case1} or \texttt{case2} is a list with only one integer member. Two such values are conversely stored in each \texttt{item[1]} within \texttt{case12}.
%-------------------------------------------------------------------------------
\begin{flushleft} \small \label{scrap7}
\protect\makebox[0ex][r]{\NWtarget{nuweb5a}{\rule{0ex}{0ex}}\hspace{1em}}$\langle\,$Compute an inverted index to reorder the vertices of Boolean arguments\nobreak\ {\footnotesize 5a}$\,\rangle\equiv$
\vspace{-1ex}
\begin{list}{}{} \item
\mbox{}\verb@@\\
\mbox{}\verb@   invertedindex = list(0 for k in range(n+m))@\\
\mbox{}\verb@   for k,item in enumerate(case1):@\\
\mbox{}\verb@      invertedindex[item[1][0]] = k@\\
\mbox{}\verb@   for k,item in enumerate(case12):@\\
\mbox{}\verb@      invertedindex[item[1][0]] = k+n1@\\
\mbox{}\verb@      invertedindex[item[1][1]] = k+n1@\\
\mbox{}\verb@   for k,item in enumerate(case2):@\\
\mbox{}\verb@      invertedindex[item[1][0]] = k+n1+n2@\\
\mbox{}\verb@@{\NWsep}
\end{list}
\vspace{-1ex}
\footnotesize\addtolength{\baselineskip}{-1ex}
\begin{list}{}{\setlength{\itemsep}{-\parsep}\setlength{\itemindent}{-\leftmargin}}
\item \NWtxtMacroRefIn\ \NWlink{nuweb3a}{3a}.
\end{list}
\end{flushleft}
%-------------------------------------------------------------------------------

%-------------------------------------------------------------------------------
\subsubsection{Re-indexing of d-cells}
%-------------------------------------------------------------------------------

\paragraph{Return the single reordered pointset and the two $d$-cell arrays}
We are now finally ready to return two reordered LAR models defined over the same set \texttt{V} of vertices, and where (a) the vertex array \texttt{V} can be written as the union of three disjoint sets of points $C_1,C_{12},C_2$; (b) the $d$-cell array \texttt{CV1} is indexed over $C_1\cup C_{12}$; (b) the $d$-cell array \texttt{CV2} is indexed over $C_{12}\cup C_{2}$. 

The \texttt{vertexRenumbering} function will return the new reordered vertex set $V = (V_1 \cup V_2) \setminus (V_1 \cap V_2)$, the two renumbered $s$-cell sets \texttt{CV1} and \texttt{CV2}, and the size \texttt{len(case12)} of $V_1 \cap V_2$.
%-------------------------------------------------------------------------------
\begin{flushleft} \small \label{scrap8}
\protect\makebox[0ex][r]{\NWtarget{nuweb5b}{\rule{0ex}{0ex}}\hspace{1em}}$\langle\,$Return the single reordered pointset and the two $d$-cell arrays\nobreak\ {\footnotesize 5b}$\,\rangle\equiv$
\vspace{-1ex}
\begin{list}{}{} \item
\mbox{}\verb@@\\
\mbox{}\verb@   V = [eval(p[0]) for p in case1] + [eval(p[0]) for p in case12] + [eval(@\\
\mbox{}\verb@            p[0]) for p in case2]@\\
\mbox{}\verb@   CV1 = [[invertedindex[v] for v in cell] for cell in CV1]@\\
\mbox{}\verb@   CV2 = [[invertedindex[v] for v in cell] for cell in CV2]@\\
\mbox{}\verb@   @\\
\mbox{}\verb@   return V, CV1, CV2, len(case12)@\\
\mbox{}\verb@@{\NWsep}
\end{list}
\vspace{-1ex}
\footnotesize\addtolength{\baselineskip}{-1ex}
\begin{list}{}{\setlength{\itemsep}{-\parsep}\setlength{\itemindent}{-\leftmargin}}
\item \NWtxtMacroRefIn\ \NWlink{nuweb3a}{3a}.
\end{list}
\end{flushleft}
%-------------------------------------------------------------------------------


\subsubsection{Example of input with some coincident vertices}
In this example we give two very simple LAR representations of 2D cell complexes, with some coincident vertices, and go ahead to re-index the vertices, according to the method implemented by the function \texttt{vertexRenumbering}.

\begin{flushleft} \small
\begin{minipage}{\linewidth} \label{scrap9}
\protect\makebox[0ex][r]{\NWtarget{nuweb6a}{\rule{0ex}{0ex}}\hspace{1em}}\verb@"test/py/boolean/test02.py"@\nobreak\ {\footnotesize 6a }$\equiv$
\vspace{-1ex}
\begin{list}{}{} \item
\mbox{}\verb@@\hbox{$\langle\,$Initial import of modules\nobreak\ {\footnotesize \NWlink{nuweb12g}{12g}}$\,\rangle$}\verb@@\\
\mbox{}\verb@@\hbox{$\langle\,$Import the module\nobreak\ ({\footnotesize \NWtarget{nuweb6b}{6b}\label{scrap10}
 }\mbox{}\verb@lar2psm@ ) {\footnotesize \NWlink{nuweb13a}{13a}}$\,\rangle$}\verb@@\\
\mbox{}\verb@@\hbox{$\langle\,$Import the module\nobreak\ ({\footnotesize \NWtarget{nuweb6c}{6c}\label{scrap11}
 }\mbox{}\verb@boolean@ ) {\footnotesize \NWlink{nuweb13a}{13a}}$\,\rangle$}\verb@@\\
\mbox{}\verb@V1 = [[1,1],[3,3],[3,1],[2,3],[2,1],[1,3]]@\\
\mbox{}\verb@V2 = [[1,1],[1,3],[2,3],[2,2],[3,2],[0,1],[0,0],[2,0],[3,0]]@\\
\mbox{}\verb@CV1 = [[0,3,4,5],[1,2,3,4]]@\\
\mbox{}\verb@CV2 = [[3,4,7,8],[0,1,2,3,5,6,7]]@\\
\mbox{}\verb@model1 = V1,CV1; model2 = V2,CV2@\\
\mbox{}\verb@VIEW(STRUCT([ @\\
\mbox{}\verb@   COLOR(CYAN)(SKEL_1(STRUCT(MKPOLS(model1)))), @\\
\mbox{}\verb@   COLOR(RED)(SKEL_1(STRUCT(MKPOLS(model2)))) ]))@\\
\mbox{}\verb@# V, CV1, CV2, n12 = vertexRenumbering(model1, model2)@\\
\mbox{}\verb@(V,CV),n1,n2,n12 = boolOps(model1,model2)@\\
\mbox{}\verb@VIEW(SKEL_1(STRUCT(MKPOLS((V,CV1)))))@\\
\mbox{}\verb@VIEW(SKEL_1(STRUCT(MKPOLS((V,CV2)))))@\\
\mbox{}\verb@@{\NWsep}
\end{list}
\vspace{-2ex}
\end{minipage}\\[4ex]
\end{flushleft}
%-------------------------------------------------------------------------------

\paragraph{Example discussion} 
The aim of the \texttt{vertexRenumbering} function is twofold: (a) eliminate vertex duplicates before entering the main part of the Boolean algorithm; (b) reorder the input representations so that it becomes less expensive to check whether a 0-cell can be shared by both the arguments of a Boolean expression, so that its coboundaries must be eventually split. Remember that for any set you have:
\[
|A\cup B| = |A|+|B|-|A\cap B|.
\]
Let us notice that in the previous example
\[
|V| = |V_1 \cup V_2| = 12 \leq |V_1|+|V_2| = 6+9 = 15,
\]
and that 
\[
|V_1|+|V_2| - |V_1 \cup V_2| = 15 - 12 = 3 = |C_{12}| = |V_1 \cap V_2|,
\]
where $C_{12}$ is the subset of vertices with duplicated instances.
%-------------------------------------------------------------------------------
\begin{flushleft} \small
\begin{minipage}{\linewidth} \label{scrap12}
\protect\makebox[0ex][r]{\NWtarget{nuweb6d}{\rule{0ex}{0ex}}\hspace{1em}}$\langle\,$Output from \texttt{test/py/boolean/test02.py}\nobreak\ {\footnotesize 6d}$\,\rangle\equiv$
\vspace{-1ex}
\begin{list}{}{} \item
\mbox{}\verb@V   = [[3.0,1.0],[2.0,1.0],[3.0,3.0],[1.0,1.0],[1.0,3.0],[2.0,3.0],@\\
\mbox{}\verb@       [3.0,2.0],[2.0,0.0],[2.0,2.0],[0.0,0.0],[3.0,0.0],[0.0,1.0]]@\\
\mbox{}\verb@CV1 = [[3,5,1,4],[2,0,5,1]]@\\
\mbox{}\verb@CV2 = [[8,6,7,10],[3,4,5,8,11,9,7]]@\\
\mbox{}\verb@@{\NWsep}
\end{list}
\vspace{-1ex}
\footnotesize\addtolength{\baselineskip}{-1ex}
\begin{list}{}{\setlength{\itemsep}{-\parsep}\setlength{\itemindent}{-\leftmargin}}
\item {\NWtxtMacroNoRef}.
\end{list}
\end{minipage}\\[4ex]
\end{flushleft}
%-------------------------------------------------------------------------------
Notice also that \texttt{V} has been reordered in three consecutive subsets $C_{1},C_{12},C_{2}$ such that \texttt{CV1} is indexed within $C_{1}\cup C_{12}$, whereas \texttt{CV2} is indexed within $C_{12}\cup C_{2}$. In our example we have  $C_{12}=\{\texttt{3,4,5}\}$: 
%-------------------------------------------------------------------------------
\begin{flushleft} \small
\begin{minipage}{\linewidth} \label{scrap13}
\protect\makebox[0ex][r]{\NWtarget{nuweb7a}{\rule{0ex}{0ex}}\hspace{1em}}$\langle\,$Reordering of vertex indexing of cells\nobreak\ {\footnotesize 7a}$\,\rangle\equiv$
\vspace{-1ex}
\begin{list}{}{} \item
\mbox{}\verb@@\\
\mbox{}\verb@>>> sorted(CAT(CV1))@\\
\mbox{}\verb@[0, 1, 1, 2, 3, 4, 5, 5]@\\
\mbox{}\verb@>>> sorted(CAT(CV2))@\\
\mbox{}\verb@[3, 4, 5, 6, 7, 7, 8, 8, 9, 10, 11]@\\
\mbox{}\verb@@{\NWsep}
\end{list}
\vspace{-1ex}
\footnotesize\addtolength{\baselineskip}{-1ex}
\begin{list}{}{\setlength{\itemsep}{-\parsep}\setlength{\itemindent}{-\leftmargin}}
\item {\NWtxtMacroNoRef}.
\end{list}
\end{minipage}\\[4ex]
\end{flushleft}
%-------------------------------------------------------------------------------
\paragraph{Cost analysis} 
Of course, this reordering after elimination of duplicate vertices will allow to perform a cheap $O(n)$ discovering of (Delaunay) cells whose vertices belong both to \texttt{V1} \emph{and} to \texttt{V2}. 
Actually, the \emph{same test} can be now used both when the vertices of the input arguments are all different, \emph{and} when they have some coincident vertices.
The total cost of such pre-processing, executed using dictionaries, is $O(n\ln n)$.


%>>>>>>>>>>>>>>>>>>>>>>>>>>>>>>>>>>>>>>>>>>>>>>>>>>>>>>>>>>>>>>>>>>>>>>>>>>>>>>>
\section{Extracting pivot $d$-cells}
%<<<<<<<<<<<<<<<<<<<<<<<<<<<<<<<<<<<<<<<<<<<<<<<<<<<<<<<<<<<<<<<<<<<<<<<<<<<<<<<

The aim of this section is to compute a (minimal) set of $d$-cells, extracted from the Delaunay complex generated by $V_1 \cup V_2$, that we call \emph{pivot cells}, that will be split using the boundary cells of both $V_1$ and $V_2$.

%-------------------------------------------------------------------------------
\subsection{Computation of Delaunay complex $\Sigma$}
%-------------------------------------------------------------------------------

The Delaunay complex is computed on the (joint) vertex set $V$, using some efficient package for dimensional-independent Delaunay computation. Here we utilize the \texttt{scipy.spatial.Delaunay} sub-package, that can be used with both 2D and 3D points. 


%-------------------------------------------------------------------------------
\subsection{Partition of $\Sigma$ into $\Sigma_\Delta$ and $\Sigma_\Omega$}
%-------------------------------------------------------------------------------




%-------------------------------------------------------------------------------
\begin{flushleft} \small
\begin{minipage}{\linewidth} \label{scrap14}
\protect\makebox[0ex][r]{\NWtarget{nuweb7b}{\rule{0ex}{0ex}}\hspace{1em}}$\langle\,$Partition of the Delaunay complex in two sub complexes\nobreak\ {\footnotesize 7b}$\,\rangle\equiv$
\vspace{-1ex}
\begin{list}{}{} \item
\mbox{}\verb@def splitDelaunayComplex(CV,n1,n2,n12):@\\
\mbox{}\verb@   def test(cell):@\\
\mbox{}\verb@      return any([v<n1 for v in cell]) and any([v>=(n1-n12) for v in cell])@\\
\mbox{}\verb@   cells_intersection, cells_union = [],[]@\\
\mbox{}\verb@   for cell in CV: @\\
\mbox{}\verb@      if test(cell): cells_intersection.append(cell)@\\
\mbox{}\verb@      else: cells_union.append(cell)@\\
\mbox{}\verb@   return cells_union,cells_intersection@\\
\mbox{}\verb@@{\NWsep}
\end{list}
\vspace{-1ex}
\footnotesize\addtolength{\baselineskip}{-1ex}
\begin{list}{}{\setlength{\itemsep}{-\parsep}\setlength{\itemindent}{-\leftmargin}}
\item \NWtxtMacroRefIn\ \NWlink{nuweb8}{8}.
\end{list}
\end{minipage}\\[4ex]
\end{flushleft}
%-------------------------------------------------------------------------------


%>>>>>>>>>>>>>>>>>>>>>>>>>>>>>>>>>>>>>>>>>>>>>>>>>>>>>>>>>>>>>>>>>>>>>>>>>>>>>>>
\section{Splitting argument chains}
%<<<<<<<<<<<<<<<<<<<<<<<<<<<<<<<<<<<<<<<<<<<<<<<<<<<<<<<<<<<<<<<<<<<<<<<<<<<<<<<
%-------------------------------------------------------------------------------
\subsection{Matching cells in $\Sigma_\Delta$ with spanning chains in $\Lambda_1$, $\Lambda_2$}
%-------------------------------------------------------------------------------
%-------------------------------------------------------------------------------
\subsection{Splitting cells}
%-------------------------------------------------------------------------------
%-------------------------------------------------------------------------------
\subsection{Keeping cell dictionaries updated}
%-------------------------------------------------------------------------------
%>>>>>>>>>>>>>>>>>>>>>>>>>>>>>>>>>>>>>>>>>>>>>>>>>>>>>>>>>>>>>>>>>>>>>>>>>>>>>>>
\section{Boolean outputs computations}
%<<<<<<<<<<<<<<<<<<<<<<<<<<<<<<<<<<<<<<<<<<<<<<<<<<<<<<<<<<<<<<<<<<<<<<<<<<<<<<<
%>>>>>>>>>>>>>>>>>>>>>>>>>>>>>>>>>>>>>>>>>>>>>>>>>>>>>>>>>>>>>>>>>>>>>>>>>>>>>>>
\section{Export the boolean module}
%<<<<<<<<<<<<<<<<<<<<<<<<<<<<<<<<<<<<<<<<<<<<<<<<<<<<<<<<<<<<<<<<<<<<<<<<<<<<<<<

The \texttt{boolean.py} module is exported to the library \texttt{lar-cc/lib}. Therefore many of the macros developed in this module are expanded and written to an external file.

%------------------------------------------------------------------
\begin{flushleft} \small
\begin{minipage}{\linewidth} \label{scrap15}
\protect\makebox[0ex][r]{\NWtarget{nuweb8}{\rule{0ex}{0ex}}\hspace{1em}}\verb@"lib/py/boolean.py"@\nobreak\ {\footnotesize 8 }$\equiv$
\vspace{-1ex}
\begin{list}{}{} \item
\mbox{}\verb@""" Module with Boolean operators using chains and CSR matrices """@\\
\mbox{}\verb@@\hbox{$\langle\,$Initial import of modules\nobreak\ {\footnotesize \NWlink{nuweb12g}{12g}}$\,\rangle$}\verb@@\\
\mbox{}\verb@@\hbox{$\langle\,$Symbolic utility to represent points as strings\nobreak\ {\footnotesize \NWlink{nuweb14}{14}}$\,\rangle$}\verb@@\\
\mbox{}\verb@@\hbox{$\langle\,$Affine transformations of $n$ points\nobreak\ {\footnotesize \NWlink{nuweb13b}{13b}}$\,\rangle$}\verb@@\\
\mbox{}\verb@@\hbox{$\langle\,$Generation of $n$ random points in the unit $d$-disk\nobreak\ {\footnotesize \NWlink{nuweb9f}{9f}}$\,\rangle$}\verb@@\\
\mbox{}\verb@@\hbox{$\langle\,$Generation of $n$ random points in the standard $d$-cuboid\nobreak\ {\footnotesize \NWlink{nuweb10a}{10a}}$\,\rangle$}\verb@@\\
\mbox{}\verb@@\hbox{$\langle\,$Triangulation of random points\nobreak\ {\footnotesize \NWlink{nuweb10b}{10b}}$\,\rangle$}\verb@@\\
\mbox{}\verb@@\hbox{$\langle\,$Boolean subdivided complex\nobreak\ {\footnotesize \NWlink{nuweb2b}{2b}}$\,\rangle$}\verb@@\\
\mbox{}\verb@@\hbox{$\langle\,$High-level boolean operations\nobreak\ {\footnotesize \NWlink{nuweb2a}{2a}}$\,\rangle$}\verb@@\\
\mbox{}\verb@@\hbox{$\langle\,$Place the vertices of Boolean arguments in a common space\nobreak\ {\footnotesize \NWlink{nuweb3a}{3a}}$\,\rangle$}\verb@@\\
\mbox{}\verb@@\hbox{$\langle\,$Show vertices of arguments\nobreak\ {\footnotesize ?}$\,\rangle$}\verb@@\\
\mbox{}\verb@@\hbox{$\langle\,$Partition of the Delaunay complex in two sub complexes\nobreak\ {\footnotesize \NWlink{nuweb7b}{7b}}$\,\rangle$}\verb@@\\
\mbox{}\verb@@{\NWsep}
\end{list}
\vspace{-2ex}
\end{minipage}\\[4ex]
\end{flushleft}
%------------------------------------------------------------------

%>>>>>>>>>>>>>>>>>>>>>>>>>>>>>>>>>>>>>>>>>>>>>>>>>>>>>>>>>>>>>>>>>>>>>>>>>>>>>>>
\section{Tests}
%<<<<<<<<<<<<<<<<<<<<<<<<<<<<<<<<<<<<<<<<<<<<<<<<<<<<<<<<<<<<<<<<<<<<<<<<<<<<<<<
%-------------------------------------------------------------------------------
\subsection{Generation of random data}

We found useful to drive the development of new modules using randomly generated data, so that every upcoming execution of the developed algorithms is naturally driven to be challenged by different data.

\subsubsection{Testing the main algorithm}

\paragraph{Write the test executable file}

%------------------------------------------------------------------
\begin{flushleft} \small
\begin{minipage}{\linewidth} \label{scrap16}
\protect\makebox[0ex][r]{\NWtarget{nuweb9a}{\rule{0ex}{0ex}}\hspace{1em}}\verb@"test/py/boolean/test01.py"@\nobreak\ {\footnotesize 9a }$\equiv$
\vspace{-1ex}
\begin{list}{}{} \item
\mbox{}\verb@""" test program for the boolean module """@\\
\mbox{}\verb@@\hbox{$\langle\,$Initial import of modules\nobreak\ {\footnotesize \NWlink{nuweb12g}{12g}}$\,\rangle$}\verb@@\\
\mbox{}\verb@@\hbox{$\langle\,$Import the module\nobreak\ ({\footnotesize \NWtarget{nuweb9b}{9b}\label{scrap17}
 }\mbox{}\verb@boolean@ ) {\footnotesize \NWlink{nuweb13a}{13a}}$\,\rangle$}\verb@@\\
\mbox{}\verb@@\hbox{$\langle\,$Import the module\nobreak\ ({\footnotesize \NWtarget{nuweb9c}{9c}\label{scrap18}
 }\mbox{}\verb@lar2psm@ ) {\footnotesize \NWlink{nuweb13a}{13a}}$\,\rangle$}\verb@@\\
\mbox{}\verb@@\hbox{$\langle\,$Import the module\nobreak\ ({\footnotesize \NWtarget{nuweb9d}{9d}\label{scrap19}
 }\mbox{}\verb@simplexn@ ) {\footnotesize \NWlink{nuweb13a}{13a}}$\,\rangle$}\verb@@\\
\mbox{}\verb@@\hbox{$\langle\,$Import the module\nobreak\ ({\footnotesize \NWtarget{nuweb9e}{9e}\label{scrap20}
 }\mbox{}\verb@larcc@ ) {\footnotesize \NWlink{nuweb13a}{13a}}$\,\rangle$}\verb@@\\
\mbox{}\verb@model1 = randomTriangulation(100,2,'disk')@\\
\mbox{}\verb@VIEW(EXPLODE(1.5,1.5,1)(MKPOLS(model1)))@\\
\mbox{}\verb@model2 = randomTriangulation(100,2,'cuboid')@\\
\mbox{}\verb@V2,CV2 = model2@\\
\mbox{}\verb@V2 = scalePoints(V2, [2,2])@\\
\mbox{}\verb@model2 = V2,CV2 @\\
\mbox{}\verb@VIEW(EXPLODE(1.5,1.5,1)(MKPOLS(model2)))@\\
\mbox{}\verb@V,CV_un, CV_int, n1,n2,n12 = boolOps(model1,model2)@\\
\mbox{}\verb@model = V,CV_int@\\
\mbox{}\verb@@\\
\mbox{}\verb@hpc0 = STRUCT([ COLOR(RED)(EXPLODE(1.5,1.5,1)(AA(MK)(V[:n1-n12]) )), @\\
\mbox{}\verb@            COLOR(CYAN)(EXPLODE(1.5,1.5,1)(AA(MK)(V[n1:]) ))#, @\\
\mbox{}\verb@            #COLOR(WHITE)(EXPLODE(1.5,1.5,1)(AA(MK)(V[n1-n12:n1]) )) @\\
\mbox{}\verb@            ])@\\
\mbox{}\verb@hpc1 = COLOR(RED)(EXPLODE(1.5,1.5,1)(MKPOLS((V,CV_un)) ))@\\
\mbox{}\verb@hpc2 = COLOR(CYAN)(EXPLODE(1.5,1.5,1)(MKPOLS((V,CV_int)) ))@\\
\mbox{}\verb@VIEW(STRUCT([hpc0, hpc1, hpc2]))@\\
\mbox{}\verb@@{\NWsep}
\end{list}
\vspace{-2ex}
\end{minipage}\\[4ex]
\end{flushleft}
%------------------------------------------------------------------



\subsubsection{Lowest-level space generation procedures}

\paragraph{Random points in unit disk} 
First we generate a  set of $n$ random points in the unit $D^d$ disk centred on the origin, to be subsequently used to generate a random Delaunay complex of variable granularity.

%------------------------------------------------------------------
\begin{flushleft} \small
\begin{minipage}{\linewidth} \label{scrap21}
\protect\makebox[0ex][r]{\NWtarget{nuweb9f}{\rule{0ex}{0ex}}\hspace{1em}}$\langle\,$Generation of $n$ random points in the unit $d$-disk\nobreak\ {\footnotesize 9f}$\,\rangle\equiv$
\vspace{-1ex}
\begin{list}{}{} \item
\mbox{}\verb@def randomPointsInUnitCircle(n=100,d=2, r=1):@\\
\mbox{}\verb@   points = random.random((n,d)) * ([2*math.pi]+[1]*(d-1))@\\
\mbox{}\verb@   return [[SQRT(p[1])*COS(p[0]),SQRT(p[1])*SIN(p[0])] for p in points]@\\
\mbox{}\verb@   ## TODO: correct for $d$-sphere@\\
\mbox{}\verb@@\\
\mbox{}\verb@if __name__=="__main__":@\\
\mbox{}\verb@   VIEW(STRUCT(AA(MK)(randomPointsInUnitCircle()))) @\\
\mbox{}\verb@@{\NWsep}
\end{list}
\vspace{-1ex}
\footnotesize\addtolength{\baselineskip}{-1ex}
\begin{list}{}{\setlength{\itemsep}{-\parsep}\setlength{\itemindent}{-\leftmargin}}
\item \NWtxtMacroRefIn\ \NWlink{nuweb8}{8}.
\end{list}
\end{minipage}\\[4ex]
\end{flushleft}
%------------------------------------------------------------------

\paragraph{Random points in the standard $d$-cuboid} 
A set of $n$ random $d$-points is then generated within the standard $d$-cuboid, i.e.~withing the $d$-dimensional interval with a vertex on the origin.

%------------------------------------------------------------------
\begin{flushleft} \small
\begin{minipage}{\linewidth} \label{scrap22}
\protect\makebox[0ex][r]{\NWtarget{nuweb10a}{\rule{0ex}{0ex}}\hspace{1em}}$\langle\,$Generation of $n$ random points in the standard $d$-cuboid\nobreak\ {\footnotesize 10a}$\,\rangle\equiv$
\vspace{-1ex}
\begin{list}{}{} \item
\mbox{}\verb@def randomPointsInUnitCuboid(n=100,d=2):@\\
\mbox{}\verb@   return random.random((n,d)).tolist()@\\
\mbox{}\verb@@\\
\mbox{}\verb@if __name__=="__main__":@\\
\mbox{}\verb@   VIEW(STRUCT(AA(MK)(randomPointsInUnitCuboid()))) @\\
\mbox{}\verb@@{\NWsep}
\end{list}
\vspace{-1ex}
\footnotesize\addtolength{\baselineskip}{-1ex}
\begin{list}{}{\setlength{\itemsep}{-\parsep}\setlength{\itemindent}{-\leftmargin}}
\item \NWtxtMacroRefIn\ \NWlink{nuweb8}{8}.
\end{list}
\end{minipage}\\[4ex]
\end{flushleft}
%------------------------------------------------------------------



\paragraph{Triangulation of random points} The Delaunay triangulation of \texttt{randomPointsInUnitCircle} is generated by the following macro.


%------------------------------------------------------------------
\begin{flushleft} \small
\begin{minipage}{\linewidth} \label{scrap23}
\protect\makebox[0ex][r]{\NWtarget{nuweb10b}{\rule{0ex}{0ex}}\hspace{1em}}$\langle\,$Triangulation of random points\nobreak\ {\footnotesize 10b}$\,\rangle\equiv$
\vspace{-1ex}
\begin{list}{}{} \item
\mbox{}\verb@from scipy.spatial import Delaunay@\\
\mbox{}\verb@def randomTriangulation(n=100,d=2,out='disk'):@\\
\mbox{}\verb@   if out == 'disk':@\\
\mbox{}\verb@      V = randomPointsInUnitCircle(n,d)@\\
\mbox{}\verb@   elif out == 'cuboid':@\\
\mbox{}\verb@      V = randomPointsInUnitCuboid(n,d)@\\
\mbox{}\verb@   CV = Delaunay(array(V)).vertices@\\
\mbox{}\verb@   model = V,CV@\\
\mbox{}\verb@   return model@\\
\mbox{}\verb@@\\
\mbox{}\verb@if __name__=="__main__":@\\
\mbox{}\verb@   from lar2psm import *@\\
\mbox{}\verb@   VIEW(EXPLODE(1.5,1.5,1)(MKPOLS(model)))@\\
\mbox{}\verb@@{\NWsep}
\end{list}
\vspace{-1ex}
\footnotesize\addtolength{\baselineskip}{-1ex}
\begin{list}{}{\setlength{\itemsep}{-\parsep}\setlength{\itemindent}{-\leftmargin}}
\item \NWtxtMacroRefIn\ \NWlink{nuweb8}{8}.
\end{list}
\end{minipage}\\[4ex]
\end{flushleft}
%------------------------------------------------------------------

%-------------------------------------------------------------------------------
%-------------------------------------------------------------------------------
\subsection{Disk saving of test data}
%-------------------------------------------------------------------------------
%-------------------------------------------------------------------------------
\subsection{Algorithm execution}
%-------------------------------------------------------------------------------
%-------------------------------------------------------------------------------
\subsection{Unit tests}
%-------------------------------------------------------------------------------


\begin{flushleft} \small
\begin{minipage}{\linewidth} \label{scrap24}
\protect\makebox[0ex][r]{\NWtarget{nuweb11a}{\rule{0ex}{0ex}}\hspace{1em}}\verb@"test/py/boolean/test03.py"@\nobreak\ {\footnotesize 11a }$\equiv$
\vspace{-1ex}
\begin{list}{}{} \item
\mbox{}\verb@""" test program for the boolean module """@\\
\mbox{}\verb@@\hbox{$\langle\,$Initial import of modules\nobreak\ {\footnotesize \NWlink{nuweb12g}{12g}}$\,\rangle$}\verb@@\\
\mbox{}\verb@@\hbox{$\langle\,$Import the module\nobreak\ ({\footnotesize \NWtarget{nuweb11b}{11b}\label{scrap25}
 }\mbox{}\verb@boolean@ ) {\footnotesize \NWlink{nuweb13a}{13a}}$\,\rangle$}\verb@@\\
\mbox{}\verb@@\hbox{$\langle\,$Import the module\nobreak\ ({\footnotesize \NWtarget{nuweb11c}{11c}\label{scrap26}
 }\mbox{}\verb@lar2psm@ ) {\footnotesize \NWlink{nuweb13a}{13a}}$\,\rangle$}\verb@@\\
\mbox{}\verb@@\hbox{$\langle\,$Import the module\nobreak\ ({\footnotesize \NWtarget{nuweb11d}{11d}\label{scrap27}
 }\mbox{}\verb@simplexn@ ) {\footnotesize \NWlink{nuweb13a}{13a}}$\,\rangle$}\verb@@\\
\mbox{}\verb@@\hbox{$\langle\,$Import the module\nobreak\ ({\footnotesize \NWtarget{nuweb11e}{11e}\label{scrap28}
 }\mbox{}\verb@larcc@ ) {\footnotesize \NWlink{nuweb13a}{13a}}$\,\rangle$}\verb@@\\
\mbox{}\verb@V1 = [[0,5],[6,5],[0,2],[3,2],[6,2],[0,0],[3,0],[3,-2],[6,-2]]@\\
\mbox{}\verb@CV1 = [[2,3,5,6],[0,1,2,3,4],[3,4,6,7,8]]@\\
\mbox{}\verb@blue = V1,CV1@\\
\mbox{}\verb@V2 = [[3,6],[7,6],[0,5],[3,5],[3,4],[7,4],[3,2],[7,2],[0,0],[3,0],[6,0],[6,2]]@\\
\mbox{}\verb@CV2 = [[0,1,3,4,5],[2,3,4,6,8,9],[6,9,10,11],[4,5,6,7,11]]@\\
\mbox{}\verb@red = V2,CV2@\\
\mbox{}\verb@@\\
\mbox{}\verb@VIEW(STRUCT([@\\
\mbox{}\verb@COLOR(RED)(EXPLODE(1.2,1.2,1)(MKPOLS(red))),@\\
\mbox{}\verb@COLOR(BLUE)(EXPLODE(1.2,1.2,1)(MKPOLS(blue)))@\\
\mbox{}\verb@]))@\\
\mbox{}\verb@@\\
\mbox{}\verb@V,CV_un, CV_int, n1,n2,n12 = boolOps(red,blue)@\\
\mbox{}\verb@CV = Delaunay(array(V)).vertices@\\
\mbox{}\verb@@\\
\mbox{}\verb@if n12==0:@\\
\mbox{}\verb@   hpc0 = STRUCT([ COLOR(RED)(EXPLODE(1.5,1.5,1)(AA(MK)(V[:n1-n12]) )), @\\
\mbox{}\verb@            COLOR(CYAN)(EXPLODE(1.5,1.5,1)(AA(MK)(V[n1:]) )) ])@\\
\mbox{}\verb@else:@\\
\mbox{}\verb@   hpc0 = STRUCT([ COLOR(RED)(EXPLODE(1.5,1.5,1)(AA(MK)(V[:n1-n12]) )), @\\
\mbox{}\verb@            COLOR(CYAN)(EXPLODE(1.5,1.5,1)(AA(MK)(V[n1:]) )), @\\
\mbox{}\verb@            COLOR(WHITE)(EXPLODE(1.5,1.5,1)(AA(MK)(V[n1-n12:n1]) )) ])@\\
\mbox{}\verb@@\\
\mbox{}\verb@hpc1 = COLOR(RED)(EXPLODE(1.5,1.5,1)(MKPOLS((V,CV_un)) ))@\\
\mbox{}\verb@hpc2 = COLOR(CYAN)(EXPLODE(1.5,1.5,1)(MKPOLS((V,CV_int)) ))@\\
\mbox{}\verb@VIEW(STRUCT([hpc0, hpc1, hpc2]))@\\
\mbox{}\verb@@{\NWsep}
\end{list}
\vspace{-2ex}
\end{minipage}\\[4ex]
\end{flushleft}
\begin{flushleft} \small
\begin{minipage}{\linewidth} \label{scrap29}
\protect\makebox[0ex][r]{\NWtarget{nuweb12a}{\rule{0ex}{0ex}}\hspace{1em}}\verb@"test/py/boolean/test04.py"@\nobreak\ {\footnotesize 12a }$\equiv$
\vspace{-1ex}
\begin{list}{}{} \item
\mbox{}\verb@""" test program for the boolean module """@\\
\mbox{}\verb@from pyplasm import *@\\
\mbox{}\verb@@\hbox{$\langle\,$Initial import of modules\nobreak\ {\footnotesize \NWlink{nuweb12g}{12g}}$\,\rangle$}\verb@@\\
\mbox{}\verb@@\hbox{$\langle\,$Import the module\nobreak\ ({\footnotesize \NWtarget{nuweb12b}{12b}\label{scrap30}
 }\mbox{}\verb@boolean@ ) {\footnotesize \NWlink{nuweb13a}{13a}}$\,\rangle$}\verb@@\\
\mbox{}\verb@@\hbox{$\langle\,$Import the module\nobreak\ ({\footnotesize \NWtarget{nuweb12c}{12c}\label{scrap31}
 }\mbox{}\verb@lar2psm@ ) {\footnotesize \NWlink{nuweb13a}{13a}}$\,\rangle$}\verb@@\\
\mbox{}\verb@@\hbox{$\langle\,$Import the module\nobreak\ ({\footnotesize \NWtarget{nuweb12d}{12d}\label{scrap32}
 }\mbox{}\verb@simplexn@ ) {\footnotesize \NWlink{nuweb13a}{13a}}$\,\rangle$}\verb@@\\
\mbox{}\verb@@\hbox{$\langle\,$Import the module\nobreak\ ({\footnotesize \NWtarget{nuweb12e}{12e}\label{scrap33}
 }\mbox{}\verb@larcc@ ) {\footnotesize \NWlink{nuweb13a}{13a}}$\,\rangle$}\verb@@\\
\mbox{}\verb@@\hbox{$\langle\,$Import the module\nobreak\ ({\footnotesize \NWtarget{nuweb12f}{12f}\label{scrap34}
 }\mbox{}\verb@largrid@ ) {\footnotesize \NWlink{nuweb13a}{13a}}$\,\rangle$}\verb@@\\
\mbox{}\verb@blue = larSimplexGrid([2,4])@\\
\mbox{}\verb@red = larSimplexGrid([4,3])@\\
\mbox{}\verb@VIEW(STRUCT([@\\
\mbox{}\verb@COLOR(RED)(EXPLODE(1.2,1.2,1)(MKPOLS(red))),@\\
\mbox{}\verb@COLOR(BLUE)(EXPLODE(1.2,1.2,1)(MKPOLS(blue)))@\\
\mbox{}\verb@]))@\\
\mbox{}\verb@@\\
\mbox{}\verb@V,CV_un, CV_int, n1,n2,n12 = boolOps(red,blue)@\\
\mbox{}\verb@CV = Delaunay(array(V)).vertices@\\
\mbox{}\verb@@\\
\mbox{}\verb@if n12==0:@\\
\mbox{}\verb@   hpc0 = STRUCT([ COLOR(RED)(EXPLODE(1.5,1.5,1)(AA(MK)(V[:n1-n12]) )), @\\
\mbox{}\verb@            COLOR(CYAN)(EXPLODE(1.5,1.5,1)(AA(MK)(V[n1:]) )) ])@\\
\mbox{}\verb@else:@\\
\mbox{}\verb@   hpc0 = STRUCT([ COLOR(RED)(EXPLODE(1.5,1.5,1)(AA(MK)(V[:n1-n12]) )), @\\
\mbox{}\verb@            COLOR(CYAN)(EXPLODE(1.5,1.5,1)(AA(MK)(V[n1:]) )), @\\
\mbox{}\verb@            COLOR(WHITE)(EXPLODE(1.5,1.5,1)(AA(MK)(V[n1-n12:n1]) )) ])@\\
\mbox{}\verb@@\\
\mbox{}\verb@hpc1 = COLOR(RED)(EXPLODE(1.5,1.5,1)(MKPOLS((V,CV_un)) ))@\\
\mbox{}\verb@hpc2 = COLOR(CYAN)(EXPLODE(1.5,1.5,1)(MKPOLS((V,CV_int)) ))@\\
\mbox{}\verb@VIEW(STRUCT([hpc0, hpc1, hpc2]))@\\
\mbox{}\verb@@{\NWsep}
\end{list}
\vspace{-2ex}
\end{minipage}\\[4ex]
\end{flushleft}
%<<<<<<<<<<<<<<<<<<<<<<<<<<<<<<<<<<<<<<<<<<<<<<<<<<<<<<<<<<<<<<<<<<<<<<<<<<<<<<<
%>>>>>>>>>>>>>>>>>>>>>>>>>>>>>>>>>>>>>>>>>>>>>>>>>>>>>>>>>>>>>>>>>>>>>>>>>>>>>>>
\appendix
\section{Utility functions}

%------------------------------------------------------------------
\begin{flushleft} \small
\begin{minipage}{\linewidth} \label{scrap35}
\protect\makebox[0ex][r]{\NWtarget{nuweb12g}{\rule{0ex}{0ex}}\hspace{1em}}$\langle\,$Initial import of modules\nobreak\ {\footnotesize 12g}$\,\rangle\equiv$
\vspace{-1ex}
\begin{list}{}{} \item
\mbox{}\verb@from pyplasm import *@\\
\mbox{}\verb@from scipy import *@\\
\mbox{}\verb@import os,sys@\\
\mbox{}\verb@@\\
\mbox{}\verb@""" import modules from larcc/lib """@\\
\mbox{}\verb@sys.path.insert(0, 'lib/py/')@\\
\mbox{}\verb@@{\NWsep}
\end{list}
\vspace{-1ex}
\footnotesize\addtolength{\baselineskip}{-1ex}
\begin{list}{}{\setlength{\itemsep}{-\parsep}\setlength{\itemindent}{-\leftmargin}}
\item \NWtxtMacroRefIn\ \NWlink{nuweb6a}{6a}\NWlink{nuweb8}{, 8}\NWlink{nuweb9a}{, 9a}\NWlink{nuweb11a}{, 11a}\NWlink{nuweb12a}{, 12a}.
\end{list}
\end{minipage}\\[4ex]
\end{flushleft}
%------------------------------------------------------------------

%------------------------------------------------------------------
\begin{flushleft} \small
\begin{minipage}{\linewidth} \label{scrap36}
\protect\makebox[0ex][r]{\NWtarget{nuweb13a}{\rule{0ex}{0ex}}\hspace{1em}}$\langle\,$Import the module\nobreak\ {\footnotesize 13a}$\,\rangle\equiv$
\vspace{-1ex}
\begin{list}{}{} \item
\mbox{}\verb@import @@1\verb@@\\
\mbox{}\verb@from @@1\verb@ import *@\\
\mbox{}\verb@@{\NWsep}
\end{list}
\vspace{-1ex}
\footnotesize\addtolength{\baselineskip}{-1ex}
\begin{list}{}{\setlength{\itemsep}{-\parsep}\setlength{\itemindent}{-\leftmargin}}
\item \NWtxtMacroRefIn\ \NWlink{nuweb6a}{6a}\NWlink{nuweb9a}{, 9a}\NWlink{nuweb11a}{, 11a}\NWlink{nuweb12a}{, 12a}.
\end{list}
\end{minipage}\\[4ex]
\end{flushleft}
%------------------------------------------------------------------

\paragraph{Affine transformations of points} Some primitive maps of points to points are given in the following, including translation, rotation and scaling of array of points via direct transformation of their coordinate.

%------------------------------------------------------------------
\begin{flushleft} \small
\begin{minipage}{\linewidth} \label{scrap37}
\protect\makebox[0ex][r]{\NWtarget{nuweb13b}{\rule{0ex}{0ex}}\hspace{1em}}$\langle\,$Affine transformations of $n$ points\nobreak\ {\footnotesize 13b}$\,\rangle\equiv$
\vspace{-1ex}
\begin{list}{}{} \item
\mbox{}\verb@def translatePoints (points, tvect):@\\
\mbox{}\verb@   return [VECTSUM([p,tvect]) for p in points]@\\
\mbox{}\verb@@\\
\mbox{}\verb@def rotatePoints (points, angle):@\\
\mbox{}\verb@   return [[COS(x),-SIN(y)] for x,y in points]@\\
\mbox{}\verb@@\\
\mbox{}\verb@def scalePoints (points, svect):@\\
\mbox{}\verb@   return [AA(PROD)(TRANS([p,svect])) for p in points]@\\
\mbox{}\verb@@{\NWsep}
\end{list}
\vspace{-1ex}
\footnotesize\addtolength{\baselineskip}{-1ex}
\begin{list}{}{\setlength{\itemsep}{-\parsep}\setlength{\itemindent}{-\leftmargin}}
\item \NWtxtMacroRefIn\ \NWlink{nuweb8}{8}.
\end{list}
\end{minipage}\\[4ex]
\end{flushleft}
%------------------------------------------------------------------

\subsection{Numeric utilities}

A small set of utilityy functions is used to transform a point representation as array of coordinates into a string of fixed format to be used as point key into python dictionaries.

%------------------------------------------------------------------
\begin{flushleft} \small
\begin{minipage}{\linewidth} \label{scrap38}
\protect\makebox[0ex][r]{\NWtarget{nuweb14}{\rule{0ex}{0ex}}\hspace{1em}}$\langle\,$Symbolic utility to represent points as strings\nobreak\ {\footnotesize 14}$\,\rangle\equiv$
\vspace{-1ex}
\begin{list}{}{} \item
\mbox{}\verb@""" TODO: use package Decimal (http://docs.python.org/2/library/decimal.html) """@\\
\mbox{}\verb@ROUND_ZERO = 1E-07@\\
\mbox{}\verb@def round_or_zero (x,prec=7):@\\
\mbox{}\verb@   """@\\
\mbox{}\verb@   Decision procedure to approximate a small number to zero.@\\
\mbox{}\verb@   Return either the input number or zero.@\\
\mbox{}\verb@   """@\\
\mbox{}\verb@   def myround(x):@\\
\mbox{}\verb@      return eval(('%.'+str(prec)+'f') % round(x,prec))@\\
\mbox{}\verb@   xx = myround(x)@\\
\mbox{}\verb@   if abs(xx) < ROUND_ZERO: return 0.0@\\
\mbox{}\verb@   else: return xx@\\
\mbox{}\verb@@\\
\mbox{}\verb@def prepKey (args): return "["+", ".join(args)+"]"@\\
\mbox{}\verb@@\\
\mbox{}\verb@def fixedPrec(value):@\\
\mbox{}\verb@   if abs(value - int(value))<ROUND_ZERO: value = int(value)@\\
\mbox{}\verb@   out = ('%0.7f'% value).rstrip('0')@\\
\mbox{}\verb@   if out == '-0.': out = '0.'@\\
\mbox{}\verb@   return out@\\
\mbox{}\verb@   @\\
\mbox{}\verb@def vcode (vect): @\\
\mbox{}\verb@   """@\\
\mbox{}\verb@   To generate a string representation of a number array.@\\
\mbox{}\verb@   Used to generate the vertex keys in PointSet dictionary, and other similar operations.@\\
\mbox{}\verb@   """@\\
\mbox{}\verb@   return prepKey(AA(fixedPrec)(vect))@\\
\mbox{}\verb@@{\NWsep}
\end{list}
\vspace{-1ex}
\footnotesize\addtolength{\baselineskip}{-1ex}
\begin{list}{}{\setlength{\itemsep}{-\parsep}\setlength{\itemindent}{-\leftmargin}}
\item \NWtxtMacroRefIn\ \NWlink{nuweb8}{8}.
\end{list}
\end{minipage}\\[4ex]
\end{flushleft}
%------------------------------------------------------------------




\bibliographystyle{amsalpha}
\bibliography{boolean}

\end{document}
