\documentclass[11pt,oneside]{article}	%use"amsart"insteadof"article"forAMSLaTeXformat
\usepackage{geometry}		%Seegeometry.pdftolearnthelayoutoptions.Therearelots.
\geometry{letterpaper}		%...ora4paperora5paperor...
%\geometry{landscape}		%Activateforforrotatedpagegeometry
%\usepackage[parfill]{parskip}		%Activatetobeginparagraphswithanemptylineratherthananindent
\usepackage{graphicx}				%Usepdf,png,jpg,orepsßwithpdflatex;useepsinDVImode
								%TeXwillautomaticallyconverteps-->pdfinpdflatex		
\usepackage{amssymb}
\usepackage[colorlinks]{hyperref}

%----macros begin---------------------------------------------------------------
\usepackage{color}
\usepackage{amsthm}

\def\conv{\mbox{\textrm{conv}\,}}
\def\aff{\mbox{\textrm{aff}\,}}
\def\E{\mathbb{E}}
\def\R{\mathbb{R}}
\def\Z{\mathbb{Z}}
\def\tex{\TeX}
\def\latex{\LaTeX}
\def\v#1{{\bf #1}}
\def\p#1{{\bf #1}}
\def\T#1{{\bf #1}}

\def\vet#1{{\left(\begin{array}{cccccccccccccccccccc}#1\end{array}\right)}}
\def\mat#1{{\left(\begin{array}{cccccccccccccccccccc}#1\end{array}\right)}}

\def\lin{\mbox{\rm lin}\,}
\def\aff{\mbox{\rm aff}\,}
\def\pos{\mbox{\rm pos}\,}
\def\cone{\mbox{\rm cone}\,}
\def\conv{\mbox{\rm conv}\,}
\newcommand{\homog}[0]{\mbox{\rm homog}\,}
\newcommand{\relint}[0]{\mbox{\rm relint}\,}

%----macros end-----------------------------------------------------------------

\title{Automatic generation of simplified buildings from 2D wire-frame
\footnote{This document is part of the \emph{Linear Algebraic Representation with CoChains} (LAR-CC) framework~\cite{cclar-proj:2013:00}. \today}
}
\author{Alberto Paoluzzi}
%\date{}							%Activatetodisplayagivendateornodate

\begin{document}
\maketitle
\nonstopmode

\begin{abstract}
In this module we develop a small library of functions to generate automatically simplified models of 3D buildings starting from 2D wire-frame drawings. The generated geometries will be exported to xGeoJson, a hierarchical data format extending GeoJson, the standard for geographical web maps, with assemblies and local rectangular coordinates. 
\end{abstract}

\tableofcontents

%===============================================================================
\section{Introduction}
%===============================================================================

A simplified 2D layout of building floors may be generated by using either a simple web UI providing only two or three interactive graphics primitives (rect and polyline, in particular) or by exporting the output of a drawing program to a standard 2D graphics data format, in particular to \texttt{SVG} (Simple Vector Graphics, the vector graphics standard for the web). This information will be used to automatically generate a simplified 3D model of the building and to export its geometry and topology to an external data format called \texttt{xGeoJson}, a customised extension of GeoJson an opensource standard for geographical information.


%===============================================================================
\section{Transformation filters}
%===============================================================================

%-------------------------------------------------------------------------------
\subsection{From \texttt{svg} to \texttt{lar}}
%-------------------------------------------------------------------------------

\paragraph{\texttt{SVG} primitives to lar}
Two \texttt{SVG} primitives are currently used to define the wire-frame layout of the building floors using either a drawing application or an interactive user interface within the browser:  \texttt{rect} (for the definition of rectangles) and \texttt{polyline} (for the definition of closed polylines). 

%-------------------------------------------------------------------------------
@D transform svg primitives to basic lar format
@{""" transform svg primitives to basic lar format """
def rects2polylines(rooms): 
	return [[[x,y],[x+dx,y],[x+dx,y+dy],[x,y+dy]] for x,y,dx,dy in rooms]
	
def polyline2lar(polylines):
	index,defaultValue = -1,-1
	Vdict,FV = dict(),[]
	for k,polyline in enumerate(polylines):
		cell = []
		for vert in polyline:
			key = vcode(vert)
			if Vdict.get(key,defaultValue) == defaultValue:
				index += 1
				Vdict[key] = index
				cell += [index]
			else: 
				cell += [Vdict[key]]
		FV += [cell]
	items = TRANS(Vdict.items())
	V = TRANS(sorted(zip(items[1],AA(eval)(items[0]))))[1]
	#FV = AA(sorted)(FV)
	EV = face2edge(FV)
	return V,FV,EV
@}
%-------------------------------------------------------------------------------


%-------------------------------------------------------------------------------
\subsection{From \texttt{lar} to \texttt{struct}}
%-------------------------------------------------------------------------------


\paragraph{Mapping of a \texttt{lar} model to a list of \texttt{lar} structures}

%-------------------------------------------------------------------------------
@D transform a lar model to a list of lar structures
@{""" transform a lar model to a list of lar structures """
def lar2Structs(model):
   V,FV = model
   return [ Struct([[[V[v] for v in cell], [range(len(cell))]]]) for cell in FV]
@}
%-------------------------------------------------------------------------------

%-------------------------------------------------------------------------------
@D Struct class pushing local origin at the bottom of the structure
@{class Struct2(Struct):
	def flatten(self): 
		structs = copy(self.body)
		structList = lar2Structs(CAT(structs.body))
		return [ absModel2relStruct(struct[0].body) for struct in structList ]
@}
%-------------------------------------------------------------------------------


%-------------------------------------------------------------------------------
@D transform an absolute lar model to a relative lar structure
@{""" transform an absolute lar model to a relative lar structure """
def absModel2relStruct(larModel):
   V,E = larModel
   Vnew = (array(V) - V[0]).tolist()
   return Struct([ t(*V[0]), (Vnew,E) ])
@}
%-------------------------------------------------------------------------------



%-------------------------------------------------------------------------------
\subsection{Assembling \texttt{struct}ures}
%-------------------------------------------------------------------------------

%-------------------------------------------------------------------------------
\subsection{Exporting to \texttt{xGeoJson}}
%-------------------------------------------------------------------------------


\paragraph{Exporting to JSON via YML a lar structure}

%-------------------------------------------------------------------------------
@D Exporting to JSON via YML
@{""" exporting to JSON via YML a lar structure """
file_handle = open(path+filename+".yml")
my_dictionary = yaml.safe_load(file_handle)
file_handle.close()
with open(path+filename+".json", 'w') as outfile:
     json.dump(my_dictionary, outfile, sort_keys=True, indent=4, ensure_ascii=False)
@}
%-------------------------------------------------------------------------------


\paragraph{print a lar structure to a geoJson file}

%-------------------------------------------------------------------------------
@D Print a lar structure to a geoJson file
@{""" print a lar structure to a geoJson file """
import yaml
import json

def printStruct2GeoJson(path,struct):
    if struct.__name__() == None:
        filename = str(id(struct))
    else: 
        filename = struct.__name__()
    theFile = open(path+filename+".yml", "w")
    print >> theFile, "---"
    print "filename =", path+filename+".yml"
    dim = checkStruct(struct.body)
    CTM, stack = scipy.identity(dim+1), []
    fathers = [filename]
    scene,fathers = printTraversal(theFile, CTM, stack, struct, [], 0, fathers,filename) 
    theFile.close()
    @< Exporting to JSON via YML @>
    return scene,fathers

@< Print a Model object in a geoJson file @>
@< Print a Mat object in a geoJson file @>
@< Print a Struct object in a geoJson file @>
@< Traverse a structure to print a geoJson file @>
@}
%-------------------------------------------------------------------------------

\paragraph{Print a \texttt{Model} object in a \texttt{xGeoJson} file}

%-------------------------------------------------------------------------------
@D Print a Model object in a geoJson file
@{""" Print a model object in a geoJson file """
def printModelObject(theFile,tabs, i,name,verts,cells,father):
    tab = "    "
    print >> theFile, "-   ","id:", name
    print >> theFile, tab, "type:", "Feature"
    print >> theFile, tab, "geometry:" 
    print >> theFile, tab+tab, "type:", "Polygon"
    print >> theFile, tab+tab, "coordinates:" 
    print >> theFile, tab+tab+tab, "verts:"
    print >> theFile, tab+tab+tab+tab, AA(eval)(AA(vcode)(verts))
    print >> theFile, tab+tab+tab, "cells:"
    print >> theFile, tab+tab+tab+tab, cells
    print >> theFile, tab, "properties:"
    print >> theFile, tab+tab, "class:", "room"
    print >> theFile, tab+tab, "parent:", father
    print >> theFile, tab+tab, "son:", i
    print >> theFile, tab+tab, "description:", name
    print >> theFile, tab+tab, "tVector:", [0, 0, 0]
    print >> theFile, tab+tab, "rVector:", [0, 0, 0]
@}
%-------------------------------------------------------------------------------

\paragraph{Print a \texttt{Mat} object in a \texttt{xGeoJson} file}

%-------------------------------------------------------------------------------
@D Print a Mat object in a geoJson file
@{""" Print a mat object in a geoJson file """
def printMatObject(theFile,tabs, theMat):
    tab = "    "
    print >> theFile, tab+tab, "tVector:", theMat.T[-1,:-1].tolist()
@}
%-------------------------------------------------------------------------------

\paragraph{Print a \texttt{Struct} object in a \texttt{xGeoJson} file}

%-------------------------------------------------------------------------------
@D Print a Struct object in a geoJson file
@{""" Print a struct object in a geoJson file """
def printStructObject(theFile,tabs, i,name,box,father):
    tab = "    "
    print >> theFile, "-   ","id:", name
    print >> theFile, tab,"type:", "Feature"
    print >> theFile, tab,"geometry:" 
    print >> theFile, tab+tab, "type:", "Polygon"
    print >> theFile, tab+tab, "box:", box
    print >> theFile, tab,"properties:"
    print >> theFile, tab+tab, "class:", "assembly"
    print >> theFile, tab+tab, "parent:", father
    print >> theFile, tab+tab, "son:", i
    print >> theFile, tab+tab, "description:", name
    print >> theFile, tab+tab, "tVector:", [0, 0, 0]
    print >> theFile, tab+tab, "rVector:", [0, 0, 0]
@}
%-------------------------------------------------------------------------------

\paragraph{Traverse a structure to print a geoJson file}

%-------------------------------------------------------------------------------
@D Traverse a structure to print a geoJson file
@{""" Traverse a structure to print a geoJson file """
def printTraversal(theFile,CTM, stack, obj, scene=[], level=0, fathers=[],father=""):
   tabs = (4*level)*" "
   for i in range(len(obj)):
      if isinstance(obj[i],Model): 
         i,verts,cells = obj[i]
         if obj[i].__name__() == None:
            name = id(obj[i])
         else: 
            name = obj[i].__name__()
         printModelObject(theFile,tabs, i,name,verts,cells,father)
         scene += [larApply(CTM)(obj[i])]
         fathers += [father]
      elif (isinstance(obj[i],tuple) or isinstance(obj[i],list)) and len(obj[i])==2:
         verts,cells = obj[i]
         name = id(obj[i])
         printModelObject(theFile,tabs, i,name,verts,cells,father)
         scene += [larApply(CTM)(obj[i])]
         fathers += [father]
      elif isinstance(obj[i],Mat): 
         printMatObject(theFile,tabs, obj[i])
         CTM = scipy.dot(CTM, obj[i])
      elif isinstance(obj[i], Struct):
         if obj[i].__name__() == None:
            name = id(obj[i])
         else: 
            name = obj[i].__name__()
         box = obj[i].box
         printStructObject(theFile,tabs, i,name,box,father)
         stack.append(CTM) 
         level += 1
         fathers.append(name)
         printTraversal(theFile,CTM, stack, obj[i], scene, level, fathers,name)
         name = fathers.pop()
         level -= 1
         CTM = stack.pop()
   return scene,fathers
@}
%-------------------------------------------------------------------------------


%===============================================================================
\section{\texttt{Iot3D} Exporting}
%===============================================================================

%-------------------------------------------------------------------------------
@O lib/py/iot3d.py
@{"""Module with automatic generation of simplified 3D buildings"""
import sys; sys.path.insert(0, 'lib/py/')
from architectural import *
@< Struct class pushing local origin at the bottom of the structure @>
@< transform svg primitives to basic lar format @>
@< transform a lar model to a list of lar structures @>
@< transform an absolute lar model to a relative lar structure @>
@< Print a lar structure to a geoJson file @>
@}
%-------------------------------------------------------------------------------


%===============================================================================
\section{Examples}
%===============================================================================

\subsection{A complete 3D building example}

In this section a complete 3D building example is developed, that starts by importing the data associated to the rect and polyline primitives of the 2D layout exported as svg file, and finishes by generating a complete 3D mock-up of a quite complex multi-floor office building.

%-------------------------------------------------------------------------------
@O test/py/iot3d/test01.py
@{"""Automatic construction of a simplified 3D building from 2D layout"""
import sys
PATH = "/Users/paoluzzi/Documents/RICERCA/sogei/edifici/"
sys.path.insert(0, PATH)

from buildings import *
from iot3d import *

# LAR models (absolute coordinates)
ala_est = larEmbed(1)(polyline2lar(rects2polylines(eastRooms) + eastTip))
ala_sud = larEmbed(1)(polyline2lar(rects2polylines(southRooms) + southTip))
ala_ovest = larEmbed(1)(polyline2lar(rects2polylines(westRooms) + westTip))
ala_nord = larEmbed(1)(polyline2lar(rects2polylines(northRooms) + northTip)) 
ascensori = larEmbed(1)(polyline2lar(elevators))
spazioComune = larEmbed(1)(polyline2lar(AA(REVERSE)(newLanding)))

# test of input consistency (flat assembly of LAR models)
pianoTipo = Struct([ala_est,ala_sud,ala_ovest,ala_nord,ascensori,spazioComune],"pianoTipo")
VIEW(EXPLODE(1.2,1.2,1.2)(MKPOLS(struct2lar(pianoTipo))))
VIEW(SKEL_1(EXPLODE(1.2,1.2,1.2)(MKPOLS(struct2lar(pianoTipo)))))

# LAR to structs
Ala_est = Struct(lar2Structs(ala_est),"Ala_est")
Ala_sud = Struct(lar2Structs(ala_sud),"Ala_sud")
Ala_ovest = Struct(lar2Structs(ala_ovest),"Ala_ovest")
Ala_nord = Struct(lar2Structs(ala_nord),"Ala_nord")
Ascensori = Struct(lar2Structs(ascensori),"Ascensori")
SpazioComune = Struct(lar2Structs(spazioComune),"SpazioComune")

model = struct2lar(Ala_est)
VIEW(EXPLODE(1.2,1.2,1.2)(MKPOLS(model)))
for k,room in enumerate(Ala_est.body):
    print room.body

# hierarchical assembly of simplest LAR models
TreAli = Struct([Ala_est,Ala_sud,Ala_ovest,Ascensori,SpazioComune],"TreAli")
VIEW(EXPLODE(1.2,1.2,1.2)(MKPOLS(struct2lar(TreAli))))
Ali_B_C_D = Struct(6*[TreAli, t(0,0,30)],"Ali_B_C_D")
VIEW(EXPLODE(1.2,1.2,1.2)(MKPOLS(struct2lar(Ali_B_C_D))))

Ala_A = Struct(4*[Ala_nord,  t(0,0,30)],"Ala_A")
ALA_A = EXPLODE(1.2,1.2,1.2)(MKPOLS(struct2lar(Ala_A)))
VIEW(EXPLODE(1.2,1.2,1.2)([ ALA_A, COLOR(BLUE)(SKEL_1(ALA_A)) ]))

Edificio = Struct([ Ala_A, Ali_B_C_D ],"Edificio")
V,FV = struct2lar(Edificio)
EDIFICIO = STRUCT(MKPOLS((V,FV)))
VIEW(STRUCT([ EDIFICIO, COLOR(BLUE)(SKEL_1(EDIFICIO)) ]))

VV = AA(LIST)(range(len(V)))
SK = SKEL_1(EDIFICIO)
VIEW(larModelNumbering(1,1,1)(V,[VV,[],FV],STRUCT([ COLOR(BLUE)(SK), EDIFICIO ]),20))

Va,EVa = struct2lar(TreAli)
Vb,EVb = struct2lar(Ala_A)
a = PROD([ SKEL_1(STRUCT(MKPOLS( ([ v[:2] for v in Va], EVa) ))), QUOTE(5*[30]) ])
b = PROD([ SKEL_1(STRUCT(MKPOLS( ([ v[:2] for v in Vb], EVb) ))), QUOTE(3*[30]) ])
glass = MATERIAL([1,0,0,0.3,  0,1,0,0.3,  0,0,1,0.3, 0,0,0,0.3, 100])
VIEW(glass(STRUCT([a,b])))

VIEW(STRUCT([ glass(STRUCT([a,b])), EDIFICIO, COLOR(BLUE)(SKEL_1(EDIFICIO)) ]))
scene,fathers = printStruct2GeoJson(PATH,pianoTipo)
scene,fathers = printStruct2GeoJson(PATH,Edificio)
@}
%-------------------------------------------------------------------------------



%===============================================================================
\appendix
%===============================================================================



\bibliographystyle{amsalpha}
\bibliography{iot3d}

\end{document}
