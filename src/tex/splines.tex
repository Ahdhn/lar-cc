\documentclass[11pt,oneside]{article}	%use"amsart"insteadof"article"forAMSLaTeXformat
\usepackage{geometry}		%Seegeometry.pdftolearnthelayoutoptions.Therearelots.
\geometry{letterpaper}		%...ora4paperora5paperor...
%\geometry{landscape}		%Activateforforrotatedpagegeometry
%\usepackage[parfill]{parskip}		%Activatetobeginparagraphswithanemptylineratherthananindent
\usepackage{graphicx}				%Usepdf,png,jpg,orepsßwithpdflatex;useepsinDVImode
								%TeXwillautomaticallyconverteps-->pdfinpdflatex		
\usepackage{amssymb}
\usepackage{hyperref}

%----macros begin---------------------------------------------------------------
\usepackage{color}
\usepackage{amsthm}

\def\conv{\mbox{\textrm{conv}\,}}
\def\aff{\mbox{\textrm{aff}\,}}
\def\E{\mathbb{E}}
\def\R{\mathbb{R}}
\def\Z{\mathbb{Z}}
\def\tex{\TeX}
\def\latex{\LaTeX}
\def\v#1{{\bf #1}}
\def\p#1{{\bf #1}}
\def\T#1{{\bf #1}}

\def\vet#1{{\left(\begin{array}{cccccccccccccccccccc}#1\end{array}\right)}}
\def\mat#1{{\left(\begin{array}{cccccccccccccccccccc}#1\end{array}\right)}}

\def\lin{\mbox{\rm lin}\,}
\def\aff{\mbox{\rm aff}\,}
\def\pos{\mbox{\rm pos}\,}
\def\cone{\mbox{\rm cone}\,}
\def\conv{\mbox{\rm conv}\,}
\newcommand{\homog}[0]{\mbox{\rm homog}\,}
\newcommand{\relint}[0]{\mbox{\rm relint}\,}

%----macros end-----------------------------------------------------------------

\title{Curves, surfaces and splines with LAR
\footnote{This document is part of the \emph{Linear Algebraic Representation with CoChains} (LAR-CC) framework~\cite{cclar-proj:2013:00}. \today}
}
\author{Alberto Paoluzzi}
%\date{}							%Activatetodisplayagivendateornodate

\begin{document}
\maketitle
\nonstopmode

\begin{abstract}
In this module we implement above LAR most of the parametric methods for polynomial and rational curves, surfaces and splines discussed in the book~\cite{Paoluzzi2003a}, and implemented in the PLaSM language and in the python package pyplasm. 
\end{abstract}

\tableofcontents

%===============================================================================
\section{Introduction}
%===============================================================================

%===============================================================================
\section{Transfinite B\'ezier}
%===============================================================================
%-------------------------------------------------------------------------------
@D Multidimensional transfinite B\'ezier
@{""" Multidimensional transfinite Bezier """
def larBezier(U,d=3):
	def BEZIER0(controldata_fn):
		N = len(controldata_fn)-1
		def map_fn(point):
			t = U(point)
			controldata = [fun(point) if callable(fun) else fun 
				for fun in controldata_fn]
			out = [0.0 for i in range(len(controldata[0]))]		
			for I in range(N+1):
				weight = CHOOSE([N,I])*math.pow(1-t,N-I)*math.pow(t,I)
				for K in range(len(out)):  out[K] += weight*(controldata[I][K])
			return out
		return (COMP([AA(COMP),DISTR]))([AA(SEL)(range(d)), map_fn])
	return BEZIER0

def larBezierCurve(controlpoints):
	dim = len(controlpoints[0])
	return larBezier(S1,dim)(controlpoints)
@}
%-------------------------------------------------------------------------------
%===============================================================================
\section{Coons patches}
%===============================================================================

%-------------------------------------------------------------------------------
@D Transfinite Coons patches
@{""" Transfinite Coons patches """
def larCoonsPatch (args):
	su0_fn , su1_fn , s0v_fn , s1v_fn = args
	def map_fn(point):
		u,v=point
		su0 = su0_fn(point) if callable(su0_fn) else su0_fn
		su1 = su1_fn(point) if callable(su1_fn) else su1_fn
		s0v = s0v_fn(point) if callable(s0v_fn) else s0v_fn
		s1v = s1v_fn(point) if callable(s1v_fn) else s1v_fn
		ret=[0.0 for i in range(len(su0))]	
		for K in range(len(ret)):
			ret[K] = ((1-u)*s0v[K] + u*s1v[K]+(1-v)*su0[K] + v*su1[K] + 
			(1-u)*(1-v)*s0v[K] + (1-u)*v*s0v[K] + u*(1-v)*s1v[K] + u*v*s1v[K])
		return ret
	return (COMP([AA(COMP),DISTR]))([[S1,S2,S3], map_fn])
@}
%-------------------------------------------------------------------------------


%===============================================================================
\section{Computational framework}
%===============================================================================
\subsection{Exporting the library}
%-------------------------------------------------------------------------------
@O lib/py/splines.py
@{""" Mapping functions and primitive objects """
@< Initial import of modules @>
@< Multidimensional transfinite B\'ezier @>
@< Transfinite Coons patches @>
@}


%===============================================================================
\section{Examples}
%===============================================================================

\paragraph{Some examples of curves}

%-------------------------------------------------------------------------------
@O test/py/splines/test01.py 
@{""" Example of Bezier curve """
import sys
""" import modules from larcc/lib """
sys.path.insert(0, 'lib/py/')
from splines import *

controlpoints = [[-0,0],[1,0],[1,1],[2,1],[3,1]]
dom = larDomain([32],'simplex')
obj = larMap(larBezierCurve(controlpoints))(dom)
VIEW(STRUCT(MKPOLS(obj)))

obj = larMap(larBezier(S1,2)(controlpoints))(dom)
VIEW(STRUCT(MKPOLS(obj)))
@}
%-------------------------------------------------------------------------------

\paragraph{Transfinite cubic surface}

%-------------------------------------------------------------------------------
@O test/py/splines/test02.py  
@{""" Example of transfinite surface """
import sys
""" import modules from larcc/lib """
sys.path.insert(0, 'lib/py/')
from splines import *

dom = larDomain([20],'simplex')
C0 = larBezier(S1,3)([[0,0,0],[10,0,0]])
C1 = larBezier(S1,3)([[0,2,0],[8,3,0],[9,2,0]])
C2 = larBezier(S1,3)([[0,4,1],[7,5,-1],[8,5,1],[12,4,0]])
C3 = larBezier(S1,3)([[0,6,0],[9,6,3],[10,6,-1]])
dom2D = larExtrude1(dom,20*[1./20])
obj = larMap(larBezier(S2,3)(AA(CONS)([C0,C1,C2,C3])))(dom2D)
VIEW(STRUCT(MKPOLS(obj)))
@}
%-------------------------------------------------------------------------------

\paragraph{Coons patch interpolating 4 boundary curves}

%-------------------------------------------------------------------------------
@O test/py/splines/test03.py  
@{""" Example of transfinite Coons surface """
import sys
""" import modules from larcc/lib """
sys.path.insert(0, 'lib/py/')
from splines import *
Su0 = larBezier(S1,3)([[0,0,0],[10,0,0]])
Su1 = larBezier(S1,3)([[0,10,0],[2.5,10,3],[5,10,-3],[7.5,10,3],[10,10,0]])
Sv0 = larBezier(S2,3)([[0,0,0],[0,0,3],[0,10,3],[0,10,0]])
Sv1 = larBezier(S2,3)([[10,0,0],[10,5,3],[10,10,0]])
dom = larDomain([20],'simplex')
dom2D = larExtrude1(dom,20*[1./20])
out = larMap(larCoonsPatch(AA(CONS)([Su0,Su1,Sv0,Sv1])))(dom2D)
VIEW(STRUCT(MKPOLS(out)))
@}
%-------------------------------------------------------------------------------



%===============================================================================
\appendix
\section{Utility functions}
%===============================================================================

\paragraph{Initial import of modules}

%-------------------------------------------------------------------------------
@D Initial import of modules
@{from pyplasm import *
from scipy import *
import os,sys
""" import modules from larcc/lib """
sys.path.insert(0, 'lib/py/')
from lar2psm import *
from simplexn import *
from larcc import *
from largrid import *
from mapper import *
@}
%-------------------------------------------------------------------------------


\bibliographystyle{amsalpha}
\bibliography{splines}

\end{document}
