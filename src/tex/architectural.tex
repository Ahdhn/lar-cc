\documentclass[11pt,oneside]{article}	%use"amsart"insteadof"article"forAMSLaTeXformat
\usepackage{geometry}		%Seegeometry.pdftolearnthelayoutoptions.Therearelots.
\geometry{letterpaper}		%...ora4paperora5paperor...
%\geometry{landscape}		%Activateforforrotatedpagegeometry
%\usepackage[parfill]{parskip}		%Activatetobeginparagraphswithanemptylineratherthananindent
\usepackage{graphicx}				%Usepdf,png,jpg,orepsßwithpdflatex;useepsinDVImode
								%TeXwillautomaticallyconverteps-->pdfinpdflatex		
\usepackage{amssymb}
\usepackage{hyperref}

%----macros begin---------------------------------------------------------------
\usepackage{color}
\usepackage{amsthm}

\def\conv{\mbox{\textrm{conv}\,}}
\def\aff{\mbox{\textrm{aff}\,}}
\def\E{\mathbb{E}}
\def\R{\mathbb{R}}
\def\Z{\mathbb{Z}}
\def\tex{\TeX}
\def\latex{\LaTeX}
\def\v#1{{\bf #1}}
\def\p#1{{\bf #1}}
\def\T#1{{\bf #1}}

\def\vet#1{{\left(\begin{array}{cccccccccccccccccccc}#1\end{array}\right)}}
\def\mat#1{{\left(\begin{array}{cccccccccccccccccccc}#1\end{array}\right)}}

\def\lin{\mbox{\rm lin}\,}
\def\aff{\mbox{\rm aff}\,}
\def\pos{\mbox{\rm pos}\,}
\def\cone{\mbox{\rm cone}\,}
\def\conv{\mbox{\rm conv}\,}
\newcommand{\homog}[0]{\mbox{\rm homog}\,}
\newcommand{\relint}[0]{\mbox{\rm relint}\,}

%----macros end-----------------------------------------------------------------

\title{LAR-ABC, a representation of architectural geometry \\
{\Large From concept of spaces, to design of building fabric, to construction simulation}
\footnote{This document is part of the \emph{Linear Algebraic Representation with CoChains} (LAR-CC) framework~\cite{cclar-proj:2013:00}. \today}
}
%\author{Alberto Paoluzzi}
%\date{}							%Activatetodisplayagivendateornodate

\begin{document}
\maketitle
\nonstopmode

\begin{abstract}
This paper discusses the application of LAR (Linear Algebraic Representation) scheme~\cite{Dicarlo:2014:TNL:2543138.2543294} to the whole architectural design process, from initial concept of spaces, to the additive manufacturing of design models, to the meshing for CAE analysis, to the detailed design of components of building fabric, to the BIM processing of quantities and costs. LAR (see, e.g. [1]) is a novel general and simple representation scheme for geometric design of curves, surfaces and solids, using simple, general and well founded concepts from algebraic topology. 
LAR supports all topological incidence structures, including enumerative (images), decompositive (meshes) and boundary (CAD) representations. It is dimension-independent, and not restricted to regular complexes. Furthermore, LAR enjoys a neat mathematical format—being based on chains, the domains of discrete integration, and cochains, the discrete prototype of differential forms, so naturally integrating the geometric shape with the supported physical properties. 
The LAR representation find his roots in the design language Plasm~\cite{Paoluzzi2003a}, and is currently embedded in python and javascript, providing the designer with powerful and simple tools for a geometric calculus of shapes. In this paper we introduce the motivation of this approach, discussing how it compares to other mixed-dimensionality representations of geometry and is supported by open-source software projects. We also discuss simple examples of use, with reference to various stages of the design process.
\end{abstract}

\newpage
\tableofcontents
\newpage

%-------------------------------------------------------------------------------
%===============================================================================
\section{Introduction}
%===============================================================================
%-------------------------------------------------------------------------------
2 column
%-------------------------------------------------------------------------------
%===============================================================================
\section{Linear Algebraic Representation}
%===============================================================================
%-------------------------------------------------------------------------------
\subsection{The representation scheme}
%-------------------------------------------------------------------------------
1 column
%-------------------------------------------------------------------------------
\subsection{Topological operations}
%-------------------------------------------------------------------------------
1 column
%-------------------------------------------------------------------------------
\subsection{Models and structures}
%-------------------------------------------------------------------------------
2 columns
%===============================================================================
\section{LAR for Architecture, Building and Construction}
%===============================================================================
%-------------------------------------------------------------------------------
\subsection{Architectural structures: the organisation of spaces}
%-------------------------------------------------------------------------------
1 column
%-------------------------------------------------------------------------------
\subsection{Building objects: components and assemblies}
%-------------------------------------------------------------------------------
1 column
%-------------------------------------------------------------------------------
\subsection{Construction process: computer simulation}
%-------------------------------------------------------------------------------
1 column
%-------------------------------------------------------------------------------
%===============================================================================
\section{Examples}
%===============================================================================
%-------------------------------------------------------------------------------
\subsection{Housing design}
%-------------------------------------------------------------------------------
1.5 column
%-------------------------------------------------------------------------------
\subsection{Quick BIM}
%-------------------------------------------------------------------------------
1 column
%-------------------------------------------------------------------------------
\subsection{Introducing time}
%-------------------------------------------------------------------------------
1.5 column
%-------------------------------------------------------------------------------
%===============================================================================
\section{The computational framework}
%===============================================================================
%-------------------------------------------------------------------------------
\subsection{ABC: a set of classes}
%-------------------------------------------------------------------------------
1 column
%-------------------------------------------------------------------------------
\subsection{Some specialised methods}
%-------------------------------------------------------------------------------
2 column
%-------------------------------------------------------------------------------
\subsection{Zero install: working in the browser}
%-------------------------------------------------------------------------------
0.5 column
%-------------------------------------------------------------------------------
%===============================================================================
\section{Conclusion}
%===============================================================================
%-------------------------------------------------------------------------------
\subsection{The state of the art}
%-------------------------------------------------------------------------------
0.5 column
%-------------------------------------------------------------------------------
\subsection{What to do next}
%-------------------------------------------------------------------------------
0.5 column

\bibliographystyle{amsalpha}
\bibliography{architectural}
1.5 column
= 10 pages

%-------------------------------------------------------------------------------
%===============================================================================
\appendix
\section{Appendix}
%===============================================================================
%-------------------------------------------------------------------------------

7/8 pages

\end{document}
