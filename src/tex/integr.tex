\documentclass[11pt,oneside]{article}	%use"amsart"insteadof"article"forAMSLaTeXformat
\usepackage{geometry}		%Seegeometry.pdftolearnthelayoutoptions.Therearelots.
\geometry{letterpaper}		%...ora4paperora5paperor...
%\geometry{landscape}		%Activateforforrotatedpagegeometry
%\usepackage[parfill]{parskip}		%Activatetobeginparagraphswithanemptylineratherthananindent
\usepackage{graphicx}				%Usepdf,png,jpg,orepsßwithpdflatex;useepsinDVImode
								%TeXwillautomaticallyconverteps-->pdfinpdflatex		
\usepackage{amssymb}
\usepackage[colorlinks]{hyperref}

%----macros begin---------------------------------------------------------------
\usepackage{color}
\usepackage{amsthm}

\def\conv{\mbox{\textrm{conv}\,}}
\def\aff{\mbox{\textrm{aff}\,}}
\def\E{\mathbb{E}}
\def\R{\mathbb{R}}
\def\Z{\mathbb{Z}}
\def\tex{\TeX}
\def\latex{\LaTeX}
\def\v#1{{\bf #1}}
\def\p#1{{\bf #1}}
\def\T#1{{\bf #1}}

\def\vet#1{{\left(\begin{array}{cccccccccccccccccccc}#1\end{array}\right)}}
\def\mat#1{{\left(\begin{array}{cccccccccccccccccccc}#1\end{array}\right)}}

\def\lin{\mbox{\rm lin}\,}
\def\aff{\mbox{\rm aff}\,}
\def\pos{\mbox{\rm pos}\,}
\def\cone{\mbox{\rm cone}\,}
\def\conv{\mbox{\rm conv}\,}
\newcommand{\homog}[0]{\mbox{\rm homog}\,}
\newcommand{\relint}[0]{\mbox{\rm relint}\,}


\def\vint{\int\!\!\int\!\!\int}
\def\sint{\int\!\!\int}
\def\isum{\sum_{i=1}^n}
\def\asum{\sum_{\alpha=0}^n}
\def\bsum{\sum_{\beta=0}^m}
\def\csum{\sum_{\gamma=0}^p}
\def\sssum{\asum\bsum\csum}
\def\XXa{\sum_{\tau\in T}  |{\bf a} \times {\bf b}| }
\def\XXb{\pmatrix{\alpha\cr  h\cr}}
\def\XXc{\pmatrix{\beta\cr k\cr}}
\def\XXd{\pmatrix{\gamma\cr m\cr}}
\def\XXe{\pmatrix{h\cr i\cr}} 
\def\XXf{\pmatrix{k\cr j\cr}}
\def\XXg{\pmatrix{m\cr l\cr}}
\def\XXh{\pmatrix{p\!\!+\!\!1\cr h\cr}}

%----macros end-----------------------------------------------------------------

\title{Finite domain integration of polynomials
\footnote{This document is part of the \emph{Linear Algebraic Representation with CoChains} (LAR-CC) framework~\cite{cclar-proj:2013:00}. \today}
}
\author{Alberto Paoluzzi}
%\date{}							%Activatetodisplayagivendateornodate

\begin{document}
\maketitle
\nonstopmode

\begin{abstract}
In order to plan or control the static/dynamic behaviour of models in CAD applications, it is often necessary to evaluate integral properties ot solid models (i.e. volume, centroid, moments of inertia, etc.). This module deals with the exact evaluation of inertial properties of homogeneous polyhedral objects. 
\end{abstract}

\tableofcontents

\section{Introduction}

A finite integration method from\cite{CattaniP-BIL1990} is developed here the computation of various order monomial integrals over polyhedral solids and surfaces in 3D space. The integration method can be used for the exact evaluation of domain integrals of trivariate polynomial forms.


\section{Integration of polynomials over polyhedral domains}
\index{Integration}
\label{sec:cattani}

Here we summarize from~\cite{CattaniP-BIL1990} an exact and symbolic solution
both to the surface and volume integration of polynomials, by using a
triangulation of the volume boundary.   The evaluation of
surface and volume integrals is achieved by transforming them into line
integrals over the boundary of every 2-simplex of a domain
triangulation.  A different approach to integration, using a
decomposition into volume elements induced by a boundary triangulation
is given in~\cite{Kajiya:84} where a closed formula for volume
integration over polyhedral volumes, by decomposing the solid into a set
of solid tetrahedra, but such a method cannot be used for surface
integrations.


\paragraph{Problem statement}\index{Integration!problem}

The finite method~\cite{CattaniP-BIL1990} to compute double and triplet integrals
of monomials over linear regular polyhedra in $\R^3$ is discussed. 
In particular, this method enables practical formulae for the
exact evaluation of integrals to be achieved:
\begin{eqnarray}
II_{S} \equiv  \sint_{S} f({\bf p})\,dS, \qquad
III_{P} \equiv  \vint_{P} f({\bf p})\,dV,\label{2} 
\end{eqnarray}
where $S$, and $P$ are linear and regular 2- or 3-polyhedra in $\R^3$,
$dS$ and $dV$ are the differential surface
and the differential volume.  The integrating function 
is a  trivariate polynomial  
\[ 
f({\bf p})
=
\asum\bsum\csum
a_{\alpha\beta\gamma} x^\alpha y^\beta z^\gamma,
\] 
where $\alpha,\beta,\gamma$ are non-negative integers.

Since the extension to $f({\bf p})$ is straightforwardly given by the
linearity of integral operator, we may focus on the calculation of
integrals of monomials:
\begin{eqnarray}
II_{S}^{\alpha\beta\gamma} \equiv  
\sint_{S} x^\alpha y^\beta z^\gamma \,dS,\qquad
III_{P}^{\alpha\beta\gamma} \equiv 
\vint_{P} x^\alpha y^\beta z^\gamma \,dV.
\label{7} 
\end{eqnarray}


\paragraph{Algorithm preview}\index{Algorithm!integration}

Surface integrals are computed as a summation of integrals over a 
triangulation of the surface.  Any triangle is mapped into the unit 
triangle in the 2-space of parameters, where integrals 
of monomials become particularly 
simple.  Then formulae for integrals over polyhedral volumes are 
given.  They are easily derived by transforming volume integrals in 
surface integrals.  It is possible to show that such integrals are 
computable in polynomial time, and that inertia moments are computable 
in $O(E)$ time, $E$ being the number of edges of the solid model of 
the integration domain.

A very important feature of the integration formulae presented here is
that they can also be used with a partial model of a polyhedron,
consisting of the collection of its face loops.  Loops are oriented
counter-clockwise if external, clockwise if internal to another loop. 
Such a  model, without explicit storage of face adjacencies, is
very frequently adopted in Computer Graphics.

In this case it is sufficient to consider any $n+1$-sided (also
unconnected or multiply connected) face as topological sum of $n-1$
oriented triangles $t_i$, with vertices $\langle v_0, v_i,
v_{i+1}\rangle$, where $1\le i\le n-1$ .  In applying formulae
(\ref{19}) or (\ref{27}) to such a set of triangles, any edge that
does not belong to the original polygon will be computed twice, in the
two opposite directions.  These contributions to the whole integral
will mutually cancel each other out, as they correspond to pairs of
line integrals evaluated along opposite paths.


\paragraph{Surface integration}\index{Surface!integration}

We call \emph{structure product}  the integral of a monomial over a 
simplicial complex.  Exact formulae for
 structure products over  n-sided polygons in 2-space,
the unit triangle in 2-space, and an arbitrary triangle in 3-space, 
are derived in the following.
Structure products  are a generalization of the usual products
and moments of inertia, that can be obtained from (\ref{7}) by assuming 
$\alpha + \beta + \gamma\leq 2$.  


\paragraph{Polygon integrals}\index{Polygon integrals}

A structure product over a polygon $\pi$ in the plane $xy$ is
\begin{equation} \label{9} 
II_\pi^{\alpha\beta} = \sint_{\pi} x^\alpha y^\beta \,dS,
\qquad \alpha,\beta\geq 0, integers.
\end{equation}
Such integrals can be exactly expressed, when
$\pi$ is a  polygon with $n$ oriented edges, as:
\begin{equation} \label{10} 
II_\pi^{\alpha\beta} = 
{1\over{\alpha \!+\!1}}
\isum\sum_{h=0}^{\alpha+1}
{\alpha+1\choose  h}
x_i^{\alpha+1-h} X_i^h
\sum_{k=0}^{\beta}
{{{\beta\choose  k}}\over{h+k+1}}
y_i^{\beta-k} Y_i^{k+1}
\end{equation}
where ${\bf p}_i=(x_i, y_i)$, $X_i=x_{i+1}-x_{i}$, $Y_i=y_{i+1}-y_{i}$
and ${\bf p}_{n+1} = {\bf p}_1$.  The derivation of the formula
(\ref{10}) is based on the application of Green's theorem and on
Newton's expression for binomial powers.


\paragraph{Unit triangle integrals}\index{Unit triangle integrals}

The general formula (\ref{10})  can be specialized
for the unit triangle 
$\tau' = \langle {\bf w}_o, {\bf w}_a, {\bf w}_b \rangle$, with vertices 
\begin{equation} \label{14} 
{\bf w}_o=(0,0),\qquad {\bf w}_a=(1,0),\qquad {\bf w}_b=(0,1),
\end{equation}
getting a very simplified expression.
With some  algebraic manipulations, we obtain\footnote{
$II_\pi^{\alpha\beta}$ is substituted, 
when referred to the unit triangle, by  the symbol
$II^{\alpha\beta}$.
}  
\begin{equation}\label{15} 
II^{\alpha\beta} = 
{1\over{\alpha +1}}
\sum_{h=0}^{\alpha+1}
{\alpha+1\choose  h}
{{(-1)^h}\over{h+\beta+1}},
\end{equation}
which reduces, for $\alpha=\beta=0$, to the area of the triangle (\ref{14}):
$II^{0 0} = 1/2$.


\paragraph{Triangle integrals}\index{Triangle!integrals}

In the following we derive the general expression for structure products
evaluated on  an arbitrary triangle 
$\tau=\langle {\bf v}_o,{\bf v}_a,{\bf v}_b \rangle$  of
the 3-space $xyz$, defined by ${\bf v}_o=(x_o,y_o,z_o)$ and
by the vectors ${\bf a}={\bf v}_a-{\bf v}_o$ and 
${\bf b}={\bf v}_b-{\bf v}_o$.
The parametric equation of its embedding plane is:
\begin{equation}\label{11} 
{\bf p} = {\bf v}_o+ u \,{\bf a}+ v\, {\bf b},
\end{equation}
where the area element is 
\begin{equation}\label{12} 
d\tau=
|J|\,du\,dv=
\bigg| {{\partial {\bf p}}\over {\partial u}} 
\times {{\partial {\bf p}}\over {\partial v}} \bigg|\,du\,dv=
|{\bf a}\times{\bf b}|\,du\,dv.
\end{equation}
A structure product over a triangle $\tau$ in 3-space can be transformed
by a coordinates transformation, as follows:
\begin{equation}\label{13} 
II_\tau^{\alpha\beta\gamma}=
\sint_\tau x^\alpha y^\beta z^\gamma \,d\tau
=
|{\bf a}\times{\bf b}|
\sint_{\tau'} {x^{\alpha}(u,v)} {y^{\beta}(u,v)} {z^{\gamma}(u,v)}\,du\,dv,
\end{equation}
where $\tau'$ is the $uv$ domain that corresponds to $\tau$ under the
coordinate transformation (\ref{11}).  In this case we have (the proof
is given in~\cite{CAD:90}):
\begin{eqnarray}
II_\tau^{\alpha\beta\gamma}&=&
|{\bf a}\times{\bf b}|
\sum_{h=0}^{\alpha}
{\alpha\choose h}
x_o^{\alpha-h}
\sum_{k=0}^{\beta}
{\beta \choose k}
y_o^{\beta-k}
\sum_{m=0}^{\gamma}
{\gamma \choose m}
z_o^{\gamma-m} \cdot \nonumber\\ & &
\cdot\sum_{i=0}^{h}
{h\choose i}
a_x^{h-i} b_x^i
\sum_{j=0}^{k}
{k\choose j}
a_y^{k-j} b_y^j
\sum_{l=0}^{m}
{m\choose l}
a_z^{m-l} b_z^l 
\,II^{\mu\nu} \label{17}  ,
\end{eqnarray}
where $\mu=(h+k+m)-(i+j+l), \nu=(i+j+l)$, and $II^{\mu\nu}$ is  a structure product  
over the triangle (\ref{14}), as given by formula (\ref{15}).
Of course the area of a triangle $\tau$
is:
\begin{equation}\label{18} 
II_\tau^{000}=\sint_\tau \,d\tau=
|{\bf a}\times{\bf b}|\,
II^{00}=
{{|{\bf a}\times{\bf b}|}\over 2}.
\end{equation}


\paragraph{Surface integrals}\index{Surface!integrals}

In conclusion, a structure product over a polyhedral surface $S$,
open or closed, is a summation  of structure products (\ref{17}) over 
the 2-simplices of a triangulation $K_2$ of $S$:
\begin{equation}\label{19} 
II_S^{\alpha\beta\gamma}=
\sint_S x^\alpha y^\beta z^\gamma \,dS=
\sum_{\tau\in K_2}
II_\tau^{\alpha\beta\gamma}.
\end{equation}


\paragraph{Volume integration}\index{Volume integration}

Let $P$ be a three-dimensional polyhedron bounded by a polyhedral
surface $\partial P$. The regularity of the integration domain and the continuity of 
the integrating function enable us to apply the divergence theorem, which can be 
briefly summarized, for
a vector field ${\bf F}={\bf F}({\bf p})$ as:
\begin{equation}\label{21} 
\vint_P \nabla\cdot {\bf F} \,dx\,dy\,dz=
\sint_{\partial P} {\bf F}\cdot{\bf n} \,dS=
\sum_{\tau\in K_2}
\sint_{\tau} {\bf F}\cdot{\bf n}_\tau \,d\tau,
\end{equation}
where ${\bf n}$ is the outward vector normal to the surface
portion $dS$, and hence ${\bf n}_\tau={{{\bf a}\times{\bf b}}/{|{\bf a}\times{\bf b}|}}$.

As the function $x^\alpha y^\beta z^\gamma$ equates the divergence of
the vector field $ {\bf F}= ({{x^{\alpha+1} y^\beta
z^\gamma}/{\alpha+1}}, 0, 0) $, an expression for
$III_P^{\alpha\beta\gamma}$ is easily derived, which depends only on
the 1-simplices of a triangulation of the domain boundary and on the
structure products over its 2-simplices.

As a matter of fact, we have:
\begin{eqnarray}
III_P^{\alpha\beta\gamma} 
&=&
\vint_P x^\alpha y^\beta z^\gamma \,dx\,dy\,dz\nonumber\\
&=&
\vint_P 
{\partial\over{\partial x}}\left(
{{1}\over{\alpha+1}}
x^{\alpha+1} y^\beta z^\gamma \right)\,dx\,dy\,dz\nonumber\\
&=&
{1\over{\alpha+1}}
\sum_{\tau'\in {K'}_2}
({\bf a}\times{\bf b})_x
\sint_{\tau'} 
x^{\alpha+1} y^\beta z^\gamma\,du\,dv.\label{26} 
\end{eqnarray}

Taking into account  equations (\ref{12}) and (\ref{13}), we can substitute the integral in 
the previous equation, getting finally:
\begin{equation}\label{27} 
III_P^{\alpha\beta\gamma} 
=
{1\over{\alpha+1}}
\sum_{\tau\in {K}_2}
\left[
{{({\bf a}\times{\bf b})_x}\over{|{\bf a}\times{\bf b}|}}
\right]_\tau
II_\tau^{\alpha+1,\beta,\gamma}
\end{equation}
where the surface integrals are evaluated by using the formula (\ref{17}).




\paragraph{Computation of inertia has linear complexity}
\index{Computation!inertia}

Surface and volume integrals over linear polyhedra are computable in 
linear time.  In particular, \emph{surface and volume integrals of 
a monomial $x^\alpha y^\beta z^\gamma$ over a linear 2-or 3-polyhedron 
are computable in $O(\alpha^3\beta^2\gamma^2 E)$ time,  $E$ being the 
number of edges of the polyhedron.  }

In fact for the surface and volume integrals it is very easy to see, 
from inspection of the given equations, that both integrals may be 
evaluated in $O(\alpha^3\beta^2\gamma^2 T)$ time, $T$ being the number 
of triangles of a minimal triangulation of the domain.  It is easy to 
show that the relation $T=2E-2F<2E$ holds between the number $T$ of 
triangles of a minimal triangulation of a polyhedron boundary and the 
numbers $E$ and $F$ of its original edges and faces respectively.  
When all triangle faces are triangular, the relation reduces to 
$T={2\over 3} E$.  


This property is very important for Computer-Aided Design and 
Ro\-bo\-tics applications: it demonstrates that the inertia tensor of a 
linear polyhedral solid is easy to compute.  It directly implies that
\emph{the inertia tensor of a linear polyhedron is computable in $O(E)$ 
time.  } 
As a matter of fact the elements of the inertia matrix of a
homogeneous object $B$, namely its \emph{mass} $M$, \emph{first moments}
$M_x$,$M_y$,$M_z$, \emph{products of inertia} $M_{xy},M_{yz},M_{zx}$,
and \emph{second moments} $M_{xx},M_{yy},M_{zz}$, can be all expressed
as
\begin{equation}\label{31} 
\rho \int_B x^\alpha y^\beta z^\gamma \,dV,
\end{equation}
where  $\rho$
is the constant density, and where $0\le\alpha+\beta+\gamma\le
2$.  Being $\alpha,\beta,\gamma$ bounded, the assertion follows from
the previous claim.


\section{Implementation}

\subsection{High-level interfaces}


\paragraph{Surface and volume integrals}
%-------------------------------------------------------------------------------
@D Surface and volume integrals
@{""" Surface and volume integrals """
def Surface(P, signed=False):
    return II(P, 0, 0, 0, signed)
def Volume(P):
    return III(P, 0, 0, 0)
@}
%-------------------------------------------------------------------------------


\paragraph{Terms of the Euler tensor}
%-------------------------------------------------------------------------------
@D Terms of the Euler tensor
@{""" Terms of the Euler tensor """
def FirstMoment(P):
    out = [None]*3
    out[0] = III(P, 1, 0, 0)
    out[1] = III(P, 0, 1, 0)
    out[2] = III(P, 0, 0, 1)
    return out

def SecondMoment(P):
    out = [None]*3
    out[0] = III(P, 2, 0, 0)
    out[1] = III(P, 0, 2, 0)
    out[2] = III(P, 0, 0, 2)
    return out

def InertiaProduct(P):
    out = [None]*3
    out[0] = III(P, 0, 1, 1)
    out[1] = III(P, 1, 0, 1)
    out[2] = III(P, 1, 1, 0)
    return out
@}
%-------------------------------------------------------------------------------




\paragraph{Vectors and covectors of mechanical interest}
%-------------------------------------------------------------------------------
@D Vectors and covectors of mechanical interest
@{""" Vectors and covectors of mechanical interest """
def Centroid(P):
    out = [None]*3
    firstMoment = FirstMoment(P)
    volume = Volume(P)
    out[0] = firstMoment[0]/volume
    out[1] = firstMoment[1]/volume
    out[2] = firstMoment[2]/volume
    return out

def InertiaMoment(P):
    out = [None]*3
    secondMoment = SecondMoment(P)
    out[0] = secondMoment[1] + secondMoment[2]
    out[1] = secondMoment[2] + secondMoment[0]
    out[2] = secondMoment[0] + secondMoment[1]
    return out
@}
%-------------------------------------------------------------------------------




\paragraph{Basic integration functions}
%-------------------------------------------------------------------------------
@D Basic integration functions
@{""" Basic integration functions """
def II(P, alpha, beta, gamma, signed):
    w = 0
    V, FV = P
    for i in range(len(FV)):
        tau = [V[v] for v in FV[i]]
        term = TT(tau, alpha, beta, gamma, signed)
        w += term
    return w

def III(P, alpha, beta, gamma):
    w = 0
    V, FV = P
    for i in range(len(FV)):
        tau = [V[v] for v in FV[i]]
        vo,va,vb = tau
        a = VECTDIFF([va,vo])
        b = VECTDIFF([vb,vo])
        c = VECTPROD([a,b])
        w += (c[0]/VECTNORM(c)) * TT(tau, alpha+1, beta, gamma)
    return w/(alpha + 1)

def M(alpha, beta):
    a = 0
    for l in range(alpha + 2):
        a += CHOOSE([alpha+1,l]) * POWER([-1,l])/(l+beta+1)
    return a/(alpha + 1)
@}
%-------------------------------------------------------------------------------



\paragraph{The main integration routine}
%-------------------------------------------------------------------------------
@D The main integration routine
@{""" The main integration routine """
def TT(tau, alpha, beta, gamma, signed=False):
	vo,va,vb = tau
	a = VECTDIFF([va,vo])
	b = VECTDIFF([vb,vo])
	sl = 0;
	for h in range(alpha+1):
		for k in range(beta+1):
			for m in range(gamma+1):
				s2 = 0
				for i in range(h+1): 
					s3 = 0
					for j in range(k+1):
						s4 = 0
						for l in range(m+1):
							s4 += CHOOSE([m, l]) * POWER([a[2], m-l]) \
								* POWER([b[2], l]) * M( h+k+m-i-j-l, i+j+l )
						s3 += CHOOSE([k, j]) * POWER([a[1], k-j]) \
							* POWER([b[1], j]) * s4
					s2 += CHOOSE([h, i]) * POWER([a[0], h-i]) * POWER([b[0], i]) * s3;
				sl += CHOOSE ([alpha, h]) * CHOOSE ([beta, k]) * CHOOSE ([gamma, m]) \
					* POWER([vo[0], alpha-h]) * POWER([vo[1], beta-k]) \
					* POWER([vo[2], gamma-m]) * s2
	c = VECTPROD([a, b])
	if not signed: return sl * VECTNORM(c)
	elif a[2]==b[2]==0.0: return sl * VECTNORM(c) * SIGN(c[2])
	else: print "error: in signed surface integration"
@}
%-------------------------------------------------------------------------------



\paragraph{Surface integration}

An exact evaluation of the surface of every cell of a cellular complex may be obtained by integration of the constant scalar field $x^0y^0z^0 = 1$ over the 2D polygonal cells of our hospital model, embedded in the subspace $z=0$. 

The integration over a generic planar polygon, either convex or non-convex, may by computed by summing the integrals over the coherently-oriented \emph{triangles} generated by the (oriented) edges of the (oriented) polygon and by its first vertex. An implicit assumption on the input: the vertices in each \texttt{FV} element (face) are counterclockwise ordered, so that the \texttt{triangles} are coherently oriented.

The \texttt{Surface} integration function is imported from the \texttt{lar-cc} module \texttt{integr}. The input LAR \texttt{model} is the triple of 2D vertices \texttt{V} and the two compressed characteristic matrices \texttt{FV} and \texttt{EV}, giving the set of vertices incident on every face and edge of the cellular decomposition, respectively. The \texttt{cochain} output is the numeric array of surface areas indexed by faces. 

%-------------------------------------------------------------------------------
@D Surface integration
@{from integr import *
""" Surface integration """
def surfIntegration(model):
    V,FV,EV = model
    V = [v+[0.0] for v in V]
    cochain = []
    for face in FV:
        triangles = AA(C(AL)(face[0]))(TRANS([face[1:-1],face[2:]]))
        P = V,triangles
        area = Surface(P,signed=True) 
        cochain += [abs(area)]
    return cochain
@}
%-------------------------------------------------------------------------------




\subsection{Exporting the \texttt{integr} module}

%-------------------------------------------------------------------------------
@O larlib/larlib/integr.py 
@{# -*- coding: utf-8 -*-
"""Module for integration of polynomials over 3D volumes and surfaces"""
from larlib import *

@< Surface and volume integrals @>
@< Terms of the Euler tensor @>
@< Vectors and covectors of mechanical interest @>
@< Basic integration functions @>
@< The main integration routine @>
@< Surface integration @>
@}
%-------------------------------------------------------------------------------



\section{Test examples}

\paragraph{Integrals on the standard triangle}
%-------------------------------------------------------------------------------
@O test/py/integr/test01.py
@{""" Integrals on the standard triangle """
from larlib import *

V = [[0,0,0],[1,0,0],[0,1,0]]
FV = [[0,1,2]]
P = (V,FV)
print II(P, 0, 0, 0, False)
@}
%-------------------------------------------------------------------------------


\paragraph{Integrals on the standard tetrahedron}
%-------------------------------------------------------------------------------
@O test/py/integr/test02.py
@{""" Integrals on the standard triangle """
from larlib import *

V = [[0.0, 0.0, 0.0], [1.0, 0.0, 0.0], [0.0, 1.0, 0.0], [0.0, 0.0, 1.0]]
FV = [[1,2,3],[0,3,2],[0,1,3],[0,1,2]]
P = (V,FV)
print Volume(P)
@}
%-------------------------------------------------------------------------------


\paragraph{Integrals on the standard 3D cube}

Here we give a \emph{triangulation} of the boundary of the unit standard cube in 3D,
positively oriented, i.e. with all the triangle normals pointing towards the
exterior of the solid cube.

Notice that inverting the boundary orientation changes the sign of the volume, 
but does not change the centroid.

%-------------------------------------------------------------------------------
@O test/py/integr/test03.py
@{""" Integrals on the standard 3D cube """
from larlib import *

V = [[0,0,0],[1,0,0],[0,1,0],[1,1,0],[0,0,1],[1,0,1],[0,1,1],[1,1,1]]

FV = [[1,0,2],[0,1,4],[2,0,4],[1,2,3],[1,3,5],[4,1,5],[3,2,6],[2,4,6],
	  [5,3,7],[3,6,7],[4,5,6],[6,5,7]]

P = (V,FV)
print Volume(P)
print Centroid(P)

""" changing the boundary orientation changes the sign of volume,
	 but not changes the centroid """

P = (V,AA(REVERSE)(FV))
print Volume(P)
print Centroid(P)
@}
%-------------------------------------------------------------------------------


\paragraph{Integrals on a non-convex 2D polyline}

Some problem appear when using the integration method~\cite{CattaniP-BIL1990} on 2D models.
In fact, the basic integration formulas refer to the three-variate monomial $x^\alpha y^\beta z^\gamma$, 
and work even for the integration over 2D domains embedded in 3D, but of course in this case they do not return
the integral values as \emph{signed} real numbers, but as positive numbers, since a positive orientation for a open 2D domain in 3D may be defined only locally.

The problem is solved by using the variable \texttt{signed}, set to \texttt{True} for non-convex polygons embedded in $z=0$,
as shown in the following example.

%-------------------------------------------------------------------------------
@O test/py/integr/test04.py
@{""" Integrals on 2D non-convex polyline """
from larlib import *

polyline = TRANS([[10,10,20,40,30,30,15,15],[10,20,30,20,10,15,15,10]])
model = polyline2lar([polyline])
VIEW(POLYLINE(polyline+[polyline[0]]))

V,FV,EV = model
tria = FV[0]
triangles = AA(C(AL)(0))(TRANS([tria[1:-1],tria[2:]]))
V = [v+[0.0] for v in V]
P = V,triangles

area = Surface(P,signed=True)
print "area =",area
@}
%-------------------------------------------------------------------------------


%-------------------------------------------------------------------------------
@O test/py/integr/test05.py
@{""" Integrals on 2D non-convex polyline """
from larlib import *

V,FV,EV = openCourt11.body[0]
tria = FV[0]
triangles = AA(C(AL)(0))(TRANS([tria[1:-1],tria[2:]]))
V = [v+[0.0] for v in V]
P = V,triangles

area = Surface(P,signed=True)
print "area =",area

P = V, AA(REVERSE)(triangles)
area = Surface(P,signed=True)
print "area =",area
@}
%-------------------------------------------------------------------------------

   

\appendix
\section{Utilities}



\bibliographystyle{amsalpha}
\bibliography{integr}



\end{document}
